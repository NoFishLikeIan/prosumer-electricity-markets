\documentclass{beamer}

\setlength{\unitlength}{1cm} 
\usetheme{Frankfurt}
\useoutertheme{infolines}


\usepackage[english]{babel}
\usepackage[utf8]{inputenc}

\usepackage{amsmath, amssymb, mathtools, bbm}
\usepackage{bm}

\usepackage{graphicx}
\usepackage{wrapfig}
\usepackage{tikz} 
\usepackage{relsize}
\usepackage{makecell}
\usepackage{booktabs}
\usepackage{subcaption}
\usepackage{float}
\usepackage{multirow} 

\usepackage[style=apa]{biblatex}
\usepackage{csquotes}

\bibliography{
    ../../../../Desktop/bibliographies/thesis,
    ../maths}


% Graphs
\usetikzlibrary{positioning, arrows.meta, calc, decorations.markings, math, matrix, fit, backgrounds}

\tikzset{fontscale/.style = {font=\relscale{#1}}}

\definecolor{prosumer}{cmyk}{0,0.816,0.408,0}

% Math commands
\newcommand{\E}{\mathbb{E}}
\newcommand{\R}{\mathbb{R}}
\newcommand{\B}{\mathbb{B}}

\newcommand{\matr}[1]{\bm{#1}}
\newcommand{\set}[1]{\left\{#1\right\}}

\newcommand{\Y}{\matr{Y}}
\newcommand{\I}{\matr{I}}
\newcommand{\G}{\matr{G}}
\newcommand{\T}{\matr{T}}

\newcommand{\V}{\mathbb{V}}

% PATHS

\newcommand{\outdiag}{../thesis/sections/diagrams}
\newcommand{\plotpath}{../../plots}


\author[Andrea Titton]{Andrea Titton\\[1ex]  {\small Dr. ir. F.O.O. Wagener \& Prof. dr. C.G.H. Diks}}
\title{Thesis defence}
\subtitle{Demand Shocks and Price Contagion in Electricity Markets}
\institute{Tinbergen Institute}
\date{30/08/2021}

\begin{document}

\AtBeginSubsection[]
{
    \begin{frame}
        \frametitle{Table of Contents}
        \tableofcontents[currentsection,currentsubsection]
    \end{frame}
}

\frame{\titlepage}

\begin{frame}
    \tableofcontents
\end{frame}

\section{Motivation}

\begin{frame}{Lack of price convergence}
    \begin{figure}
        \includegraphics[height = 0.7\textheight]{figures/convergence.PNG}
        \caption{\cite{Report2019}}
    \end{figure}
\end{frame}

\begin{frame}{Empirical: \citeauthor{Bockers2014} (\citeyear{Bockers2014})}
    \textbf{Empirical analysis of convergence between 2004 and 2011}

    \begin{itemize} \setlength\itemsep{1.5em}
              \pause \item Only partial convergence (mostly Germany-Austria)
              \pause \item Lack of convergence is not due to supply shocks
              \pause \item No theoretical model of \textit{effects of shocks in some countries on energy markets in neighboring countries.}
    \end{itemize}

\end{frame}

\begin{frame}{Theoretical: \citeauthor{Gebhardt2013} (\citeyear{Gebhardt2013})}
    \textbf{Why is there no convergence in EU local electricity prices?}

    \begin{itemize} \setlength\itemsep{1.5em}
              \pause \item Rational producers and traders
              \pause \item Information asymmetries between local markets
              \pause \item Traders do not engage in trade
              \pause \item \textit{our results highlight an additional advantage of ``market coupling``, i.e., integrating the national wholesale markets}
              \pause \item \textit{we assumed that (...) cross-border trades do not influence the spot prices}
    \end{itemize}
\end{frame}

\begin{frame}{This paper}
    \begin{itemize} \setlength\itemsep{1.5em}
        \item Link between cross-border and local prices (network structure)
        \item Agent based model (bounded rationality)
        \item Role of demand shocks (prosumers)
    \end{itemize}
\end{frame}

\section{Overview}

\begin{frame}{Research question}

    \begin{itemize} \setlength{\itemsep}{1em}
        \item How do exogenous and unexpected demand shocks affect the price formation mechanism? \pause
        \item How do shock in prices transmit to the cross-border market? \pause
        \item Can this shock produce sustained price imbalances? \pause
        \item Are current policies effective in this framework?
    \end{itemize}

\end{frame}

\begin{frame}{What is new?}

    Comparison with previous ABM (\cite{Weidlich2008})
    \vfill

    \visible<1->{\begin{minipage}{0.4\textwidth}
            \resizebox{\textwidth}{!}{\tikzstyle{basic} = [
draw,circle,
minimum size=10pt,
]

\tikzstyle{core} = [
draw,circle,
minimum size=10pt,
prosumer,
fill=prosumer,
text=black
]

\begin{tikzpicture}[-{Latex[scale=1]}, thick]

    \node [core] (1) {Consumers};
    \node [basic, above right = 0.5cm and 2cm of 1] (2-1) {$demand = c$};
    \node [basic, below right = 0.5cm and 2cm of 1] (2-2) {$\V(\text{supply})$};
    \node [basic, below = 1cm of 2-2] (4) {\makecell[c]{Market \\ power}};
    \node [core, left = 2cm of 4] (3) {\makecell[c]{Network \\ providers}};
    \node [basic, above right = 0.5cm and 2cm of 2-2] (5) {Prices};


    \path
    (1) edge (2-1)
    (4) edge (2-2)
    (3) edge (4)
    (2-2) edge (5)
    (2-1) edge [dashed] (5)
    ;

\end{tikzpicture}}
        \end{minipage}}
    \hfill
    \visible<2->{\begin{minipage}{0.55\textwidth}
            \resizebox{\textwidth}{!}{\tikzstyle{basic} = [
draw,circle,
minimum size=10pt
]

\tikzstyle{core} = [
draw,circle,
minimum size=10pt,
prosumer,
fill=prosumer,
text=black
]


\begin{tikzpicture}[-{Latex[scale=1]}, thick]

    \node [core] (1) {Prosumers};
    \node [core, below = 4cm of 1] (3) {Producers};
    \node [basic, right = 1cm of 1] (2-1) {Demand};
    \node [basic, right = 1cm of 3] (4) {\makecell[c]{Market \\ power}};
    \node [basic, above = 1cm of 4] (2-2) {Supply};
    \node [basic, below right = 1cm and 2cm of 2-1] (5) {Prices};
    \node [basic, right = 1cm of 4] (7) {\makecell[c]{Bargaining \\ power}};
    \node [core, right = 1cm of 7] (6) {Providers};


    \path
    (1) edge (2-1)
    (4) edge [dashed, -] (2-2)
    (3) edge (4)
    (2-2) edge (5)
    (2-1) edge (5)
    (6) edge (7)
    (7) edge (2-2) 
    ;

\end{tikzpicture}}
        \end{minipage}}
\end{frame}

\begin{frame}{Results}
    \begin{itemize} \setlength\itemsep{1.5em}
              \pause \item Bargaining power determines price contagion (bigger and more ``central'' markets)
              \pause \item Demand shocks cause \textbf{hysteresis}: sustained price imbalances - implications for prosumers
              \pause \item This implies: markets integration is not sufficient
    \end{itemize}
\end{frame}

\section{Model}

\begin{frame}{Model outline}
    \centering
    \resizebox{\textwidth}{!}{\tikzstyle{var} = [
draw,circle,
prosumer,
fill=prosumer,
text=black,
minimum size=10pt]

\tikzstyle{agent} = [
draw, circle,
fill=yellow,
minimum size=10pt]

\tikzstyle{derived} = [
draw, circle, dashed,
minimum size=10pt]

\tikzstyle{time} = [
draw=gray, rectangle,
dashed,
thick,
inner sep=5pt]

\tikzstyle{market} = [
draw=gray, rectangle,
dashed,
thick,
inner sep=15pt]


\begin{tikzpicture}[-{Latex[scale=1]}, thick]

    \node [agent] (producer_prev) {$\text{Prod}_{t-1}$};
    \node [var, below right = 1cm and 5cm of producer_prev] (s) {\makecell[c]{Elect. \\ $s_t$}};
    \node [var, right = 2cm of s] (s_prime) {$s_{t+1}$};
    \node [derived, below = 1cm of s] (demand) {\makecell[c]{Demand \\ $X_t$}};
    \node [derived, left = 1cm of demand] (producer_payoff) {\makecell[c]{Prod. payoff \\ $\pi_t$}};
    \node [var, below = 1cm of demand] (price) {$p_t$};
    \node [agent, below right = 0.5cm and 1cm of price] (provider) {Prov.};
    \node [derived, right = 1cm of demand] (provider_payoff) {\makecell[c]{Prov. payoff \\ $\Pi_t$}};


    \path
    (s) edge [dotted] node [above] {$\gamma$} (s_prime)
    (producer_prev) edge [] node [var] (ramp) {$r_{t-1}$} (s)
    (provider) edge (price)
    (price) edge (demand)
    (demand) edge (producer_payoff)
    (demand) edge (provider_payoff)
    (s) edge (demand)
    (producer_payoff) edge [-, bend left, dashed] node [fill=white] {$\E_{\psi}$}(producer_prev)

    (provider_payoff) edge [-, dashed] node [fill=white] {$\E$}(provider)

    (ramp) edge [] node [below left] {$c(r)$} (producer_payoff)
    ;

    \node [time, fit = (producer_prev) (ramp), label=above:{Previous period}] (Previous) {};
    \node [market, fit = (producer_prev) (s_prime) (s) (Previous), label=above:{Production}] (Production) {};
    \node [market, fit = (provider) (price), label=below:{Provider}] (Provider) {};

    \matrix [draw,below left, fill=white] at (current bounding box.north east) {
    \node [agent,label=right:{Agents}] {}; \\
    \node [derived,label=right:{Derived values}] {}; \\
    \node [var,label=right:{Variables}] {}; \\
    };

\end{tikzpicture}}
\end{frame}

\begin{frame}
    Prosumers...
    \begin{itemize} \setlength\itemsep{1.5em}
              \pause \item ...demand $e_t$ which can be either low ($\underline{e}$) or high ($\overline{e}$)
              \pause \item ...a ``demand shock'' is a period of $\overline{e}$
    \end{itemize}
\end{frame}


\begin{frame}
    \begin{columns}[T,onlytextwidth]

        \begin{column}{.35\textwidth}
            Producers...
            \begin{itemize} \setlength\itemsep{1.5em}
                      \pause \item ...make delayed production decision $s_{t+1} = s_t + \underbrace{r_t(s_t, p_t)}_{\text{policy}}$
                      \pause \item ...believe price to be constant $\B_t[p_{t+1}] = p_t$
                      \pause \item ...unit production cost $k$ and marginal cost of ramp-up $\max\{0, s_t \ r_t\}$
            \end{itemize}
        \end{column}

        \hfill

        \begin{column}{.64\textwidth}
            \begin{figure}
                \includegraphics[width=\linewidth]{../../plots/rfunction.pdf}
            \end{figure}
        \end{column}
    \end{columns}
\end{frame}

\begin{frame}
    \visible<1->{
        Local markets can have \textbf{excess demand} $X$ which evolves as,

        \begin{equation*}
            X_{t+1} - X_t = \overbrace{M (e_{t+1} - e_t)}^{\text{change in demand}} - \overbrace{R_t}^{\text{change in supply }\sum_{i, t} r_{i, t}}
        \end{equation*}
    }

    \visible<2->{
        Providers form believes over

        \begin{equation*}
            \B_t \left[R_t(p_t, S_t) \right] = \alpha_t + \gamma_t \ \overbrace{p_t}^{\text{local price}} + \eta_t \ \overbrace{S_t}^{\text{local supply}}
        \end{equation*} updated via WLS.
    }
\end{frame}



% CB market
\begin{frame}{Providers: link between local and cross-border markets}
    \centering
    \resizebox{\textwidth}{!}{\tikzstyle{var} = [
draw,circle,
prosumer,
fill=prosumer,
text=black,
minimum size=10pt]

\tikzstyle{agent} = [
draw, circle,
fill=blue!30,
minimum size=10pt]

\tikzstyle{onnode} = [
    draw, circle, fill=white
]

\tikzstyle{time} = [
draw=gray, rectangle,
dashed,
thick,
inner sep=5pt]

\tikzstyle{market} = [
draw=gray, circle,
dashed,
thick,
inner sep=5pt]

\begin{tikzpicture}[-{Latex}, thick, every text node part/.style={align=center, fontscale=0.8}]

    % Placing providers
    \foreach \x/\y [count=\j] in {1/1, -1/1, 1/-1, -1/-1}
        % Draw provider
        {
            \node [agent]  (\j) at (2*\x, 2*\y) [] {Provider \j};
            \node [market] (\j-mkt) at (5*\x, 5*\y) [] {Wholesale \\ market \j};
            \path 
                (\j-mkt) edge [bend right] node [onnode] {$X_{\j}$} (\j)
                (\j) edge [dashed, bend right] node [onnode, dashed] {$p_{\j}$} (\j-mkt)
            ;
        }

    \path
    (1) edge [] node [above] {$Y^{(1, 2)}$} (2)
    (1) edge [] node [right] {$Y^{(1, 3)}$} (3)
    (3) edge [] node [above] {$Y^{(3, 4)}$} (4);

\end{tikzpicture}}
\end{frame}

\begin{frame}

    The optimization problem determines the \textbf{local price evolution}
    \begin{equation*}
        p_{i, t+1} - p_{i, t} = \underbrace{L_{i, t}(X_{i, t}, S_{i, t})}_{\text{optimization in local market}} + \overbrace{2 P(\Y_t)}^{\text{influence of cross-border market}}
    \end{equation*}

    \begin{itemize} \setlength\itemsep{1.5em}
        \item Cross-border and local markets link (addressing \citeauthor{Gebhardt2013})
        \item Determines: shock of neighbor $\xrightarrow{}$ local prices (addressing \citeauthor{Bockers2014})
    \end{itemize}

\end{frame}

\begin{frame}{Local policy: $L_{i, t}(X_{i, t}, S_{i, t})$}
    $L_{i, t}$ for ``true'' beliefs $(\alpha_t, \gamma_t, \eta_t)$
    \begin{figure}
        \includegraphics[width=0.7\linewidth]{../../plots/pricingnonbinding.pdf}
    \end{figure}
\end{frame}

\begin{frame}{Nash-bargaining in the cross-border market: $P(\Y_t)$}
    Given $X_{i, t}$ and $p_{i, t}$, Nash-bargaining solution of trading price

    \begin{equation*}
        P_t^{(i, j)} = \arg \max \{  \text{joint profits of $i$ and $j$} \}
    \end{equation*}

    Depends on

    \begin{itemize} \setlength\itemsep{1.5em}
        \item Revenue difference $\Delta^{(i, j)}_t = X_{i, t} p_{i, t} - X_{j, t} p_{j, t} $
        \item Outside options of $i$ $\sum_m Y_t^{(i, m)} P_t^{(i, m)}$ and $j$ $\sum_l Y_t^{(j, l)} P_t^{(j, l)}$
    \end{itemize}

\end{frame}

\begin{frame}

    \begin{equation*}
        P_t^{(i, j)} = \frac{1}{2 \ Y_t^{(i, j)}}  \left(\Delta^{(i, j)}_t + \underbrace{\sum_{m} Y_t^{(j, m)} \  P_t^{(j, m)}}_{\text{outside option of } j} - \underbrace{\sum_{l} Y_t^{(i, l)} \  P_t^{(i, l)}}_{\text{outside option of } i} \right)
    \end{equation*}


    Takeaways for demand shocks...

    \begin{itemize} \setlength\itemsep{1.5em}
              \pause \item Contemporaneous: $X_{j, t} \xrightarrow{} \Delta^{(i, j)}_t \xrightarrow{} P_t^{(i, j)} \xrightarrow{} p_{i, t}$
              \pause \item Delayed: $p_{i, t}  \xrightarrow{} X_{i, t+1} \xrightarrow{} \Y_t$
              \pause \item Network structure of outside options
    \end{itemize}

\end{frame}


\begin{frame}{Recap}
    \renewcommand{\arraystretch}{3}
    \resizebox{\linewidth}{!}{
        \begin{tabular}{c  c | c }
            Agent            & Actual process                                     & Perceived process                                                                 \\
            \midrule
            \boxed{Provider} & $R_t = \sum^N_{i = 1} r(s_{i, t}, p_t)$            & $ \B_{t} \left[R_{t+1} \right] = \alpha_{t} + \gamma_{t} \  p_t + \eta_{t} \ S_t$ \\
            \midrule
            \boxed{Producer} & $p_{t+1} = p_t + L_t(X_t, S_t) + 2P(\matr{Y}_{t})$ & $\B\left[p_{t+1}\right] = p_t$
        \end{tabular}
    }
\end{frame}

\section{Contagion}

\begin{frame}{Bargaining influence matrix}

    Stacking in vectors,

    \begin{equation*}
        \begin{split}
            2(P_t \circ \Y_t) &= \Delta_t  - \G \left( P_t \circ \Y_t \right) \\
            (2\I + \G) (P_t \circ \Y_t) &= \Delta_t  \\
            (P_t \circ \Y_t) &= (2\I + \G)^{-1} \Delta_t
        \end{split}
    \end{equation*}

    Contagion is determined, via $P(\Y_t)$, by

    \begin{equation*}
        (2\I + \G)^{-1}
    \end{equation*}

    called here \textit{bargaining influence matrix}

\end{frame}

\begin{frame}{Star}

    If $\mathcal{A}$ (\textit{left}) is a star graph then $L(\mathcal{A})$ (\textit{right}) is a complete graph. \vspace{5mm}

    \begin{columns}[T,onlytextwidth]

        \begin{column}{.46\textwidth}
            \resizebox{\linewidth}{!}{\tikzstyle{var} = [
draw,circle,
minimum size=10pt]

\tikzstyle{agent} = [
draw, circle,
minimum size=10pt]

\begin{tikzpicture}[-{Latex[scale=1]}, thick]

    \node [agent] (one) {Prov. $1$};
    \node [agent, left = 3cm of one] (two) {Prov. $2$};
    \node [agent, above = 3cm of one] (three) {Prov. $3$};
    \node [agent, right = 3cm of one] (four) {Prov. $4$};
    \node [agent, below = 3cm of one] (n) {Prov. $n$};


    \path
    (one) edge [] node [above] {$Y^{(1, 2)}$} (two)
    (one) edge [] node [left] {$Y^{(1, 3)}$} (three)
    (one) edge [] node [above] {$Y^{(1, 4)}$} (four)
    (one) edge [] node [right] {$Y^{(1, n)}$} (n);

\end{tikzpicture}}
        \end{column}

        \hfill

        \begin{column}{.46\textwidth}
            \resizebox{\linewidth}{!}{\tikzstyle{var} = [
draw,circle,
minimum size=10pt]

\tikzstyle{agent} = [
draw, circle,
minimum size=10pt]

\begin{tikzpicture}[-, thick]

    \node [agent] (one-two) {$(1, 2)$};
    \node [agent, above right = 1cm and 1cm of one-two] (one-three) {$(1, 3)$};
    \node [agent, below right = 1cm and 1cm of one-three] (one-four) {$(1, 4)$};
    \node [agent, below right = 1cm and 1cm of one-two] (one-n) {$(1, n)$};


    \path
    (one-two) edge [] node {} (one-three)
    (one-two) edge [] node {} (one-four)
    (one-three) edge [] node {} (one-four)
    (one-two) edge [] node {} (one-n)
    (one-three) edge [] node {} (one-n)
    (one-four) edge [] node {} (one-n);

\end{tikzpicture}}
        \end{column}
    \end{columns}
\end{frame}

\begin{frame}{Path}
    If $\mathcal{A}$ (\textit{left}) is a path graph then $L(\mathcal{A})$ (\textit{right}) is a path graph. \vspace{5mm}

    \begin{columns}[T,onlytextwidth]

        \begin{column}{.46\textwidth}
            \resizebox{\linewidth}{!}{\tikzstyle{var} = [
draw,circle,
minimum size=10pt]

\tikzstyle{agent} = [
draw, circle,
minimum size=10pt]

\begin{tikzpicture}[-, thick]

    \node [agent] (one) {Prov. $1$};
    \node [agent, right = 3cm of one] (two) {Prov. $2$};
    \node [agent, right = 3cm of two] (three) {Prov. $3$};
    \node [agent, right = 3cm of three] (four) {Prov. $n$};


    \path
    (one) edge [] node [above] {$Y^{(1, 2)}$} (two)
    (two) edge [] node [above] {$Y^{(2, 3)}$} (three)
    (three) edge [dashed] node [above] {} (four);

\end{tikzpicture}}
        \end{column}

        \hfill

        \begin{column}{.46\textwidth}
            \resizebox{\linewidth}{!}{\tikzstyle{var} = [
draw,circle,
minimum size=10pt]

\tikzstyle{agent} = [
draw, circle,
minimum size=10pt]

\begin{tikzpicture}[-{Latex[scale=1]}, thick]

    \node [agent] (one-two) {$(1, 2)$};
    \node [agent, below right = 2cm and 2cm of one-two] (two-three) {$(2, 3)$};
    \node [agent, below right = 2cm and 2cm of two-three] (three-four) {$(3, 4)$};
    \node [agent, below right = 2cm and 2cm of three-four] (last) {$(n, n-1)$};


    \path
    (one-two) edge [] node [below left] {Prov. $2$} (two-three)
    (two-three) edge [] node [below left] {Prov. $3$} (three-four)
    (three-four) edge [dashed] (last);

\end{tikzpicture}}
        \end{column}
    \end{columns}
\end{frame}


\begin{frame} {Star: from $(2\I + \G)^{-1}$ to $p_{t+1}$ }

    \begin{columns}
        \begin{column}{.47\textwidth}
            \begin{figure}
                \includegraphics[width = \linewidth]{../../plots/bargmatrices/star.pdf}
            \end{figure}
        \end{column}
        \begin{column}{.47\textwidth}
            \begin{figure}
                \includegraphics[width = \linewidth]{../../plots/pricing.pdf}
            \end{figure}
        \end{column}
    \end{columns}

\end{frame}

\begin{frame} {Path: from $(2\I + \G)^{-1}$ to $p_{t+1}$ }
    \begin{columns}
        \begin{column}{.47\textwidth}
            \begin{figure}
                \includegraphics[width = \linewidth]{../../plots/bargmatrices/path.pdf}
            \end{figure}
        \end{column}
        \begin{column}{.47\textwidth}
            \begin{figure}
                \includegraphics[width = \linewidth]{../../plots/pricingpath.pdf}
            \end{figure}
        \end{column}
    \end{columns}
\end{frame}

\begin{frame}{Star: central demand shock simulation}
    \begin{figure}[H]
        \centering
        \includegraphics[height = 0.8\textheight]{\plotpath/central/star/pricesupply.pdf}
    \end{figure}

\end{frame}

\begin{frame} {Path: central demand shock simulation}
    \begin{figure}[H]
        \centering
        \includegraphics[height = 0.8\textheight]{\plotpath/central/path/pricesupply.pdf}
    \end{figure}
\end{frame}

\begin{frame} {Excess demand $X_{i, t}$ in simulation}
    \begin{center}
        \begin{figure}[H]
            \begin{subfigure}{0.475\textwidth}
                \centering
                \includegraphics[width = \textwidth]{\plotpath/central/star/demand.pdf}
                \caption{In a star graph} \label{fig:demandshock_star}
            \end{subfigure} \hfill
            \begin{subfigure}{0.475\textwidth}
                \centering
                \includegraphics[width = \textwidth]{\plotpath/central/path/demand.pdf}
                \caption{In a path graph} \label{fig:demandshock_path}
            \end{subfigure}
        \end{figure}
    \end{center}
\end{frame}

\section{Conclusion}

\begin{frame}{Takeaways}
    \begin{itemize} \setlength\itemsep{1.5em}
        \item Market power and rational expectations are not adequate frameworks \pause
        \item Demand shocks can create sustained imbalances (prosumers) \pause
        \item Further market integration needs to address ``bargaining power''
    \end{itemize}
\end{frame}

\begin{frame}{What's next}
    \begin{itemize} \setlength\itemsep{1.5em}
        \item Further questions: \pause
              \begin{itemize} \setlength\itemsep{1em}
                  \item What if prosumers sell electricity ($e_t < 0$)? \pause
                  \item What if markets have different sizes ($M$ and $N$)? \pause
              \end{itemize}
        \item Model improvements: \pause
              \begin{itemize} \setlength\itemsep{1em}
                  \item Bipartite graphs of buyers and seller \pause
                  \item Simplify producers further and focus on providers bargaining procedure \pause
              \end{itemize}
        \item Model calibration
    \end{itemize}
\end{frame}


\begin{frame}[allowframebreaks]{Bibliography}
    \printbibliography
\end{frame}

\end{document}