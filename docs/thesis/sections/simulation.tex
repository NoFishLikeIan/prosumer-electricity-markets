\documentclass[../thesis.tex]{subfiles}

\section{Simulation} \label{a:initsim}

The code for the simulation, including all of its iterations, is hosted at \url{github.com/NoFishLikeIan/prosumer-electricity-markets}. It requires at least \verb_Julia 1.5.2_. and the package \verb_Agents.jl_ (\cite{Agents.jl})

Parameter choice in agent based models is a notoriously tricky matter (see for example \cite{Fagiolo2007}). In this case I focused simply on picking a stable set of parameters and ran a set of robustness checks. The relevant parameters are,

\begin{equation}
    \set{N, M, k, \beta, e_t, e_{\tau}}.
\end{equation}

All but $\beta$ simply change the scale of the problem. For example increasing market size $M$ in all markets simply increases the demand. As long as there is convergence, the system will satisfy that demand in the steady state. On the other hand $\beta$ has implications for the policy function of the agents. In Figure (\ref{fig:beta_stable}) I simulate the two providers case with a range of values of $\beta \in [0.9, 0.99]$. The qualitative feature of the model remain invariant, yet for lower values of $\beta$, the model is knocked off of stability sooner. To err on the side of caution I pick the most stable value of $\beta$, namely $0.99$.

\begin{figure}[H]
    \centering
    \includegraphics[width = \textwidth]{\plotpath/stability/stability.pdf}
    \caption{Price simulation in two provider case with different values of $\beta$}
    \label{fig:beta_stable}
\end{figure}

The parameter set used for most of the simulations is reported in Table \ref{table:parameters}.

\begin{table}[H]
    \centering
    \renewcommand{\arraystretch}{1.5}

\vspace{5mm}
\begin{tabular}{|| c | c ||}
    \textbf{Parameters} & \textbf{Value in simulation} \\
    $\beta$             & 0.99                         \\
    $e_t$               & 2                            \\
    $e_\tau$            & 3.2                          \\
    $N$                 & 50                           \\
    $M$                 & 1000                         \\
    $k$                 & 1
\end{tabular}
\vspace{5mm}

    \caption{Simulation parameters}
    \label{table:parameters}
\end{table}