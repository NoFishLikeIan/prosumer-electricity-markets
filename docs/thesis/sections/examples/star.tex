A more interesting example is that of the star graph, namely each provider has as a neighbor provider 1. Assume there are 4 providers/markets as below.

\vspace{5mm}
\begin{minipage}{.5\textwidth}
    \resizebox{\textwidth}{!}{\tikzstyle{var} = [
draw,circle,
minimum size=10pt]

\tikzstyle{agent} = [
draw, circle,
minimum size=10pt]

\begin{tikzpicture}[-{Latex[scale=1]}, thick]

    \node [agent] (one) {Prov. $1$};
    \node [agent, left = 3cm of one] (two) {Prov. $2$};
    \node [agent, above = 3cm of one] (three) {Prov. $3$};
    \node [agent, right = 3cm of one] (four) {Prov. $4$};


    \path
    (one) edge [] node [above] {$Y^{(1, 2)}$} (two)
    (one) edge [] node [left] {$Y^{(1, 3)}$} (three)
    (one) edge [] node [above] {$Y^{(1, 4)}$} (four);

\end{tikzpicture}}
\end{minipage}
\begin{minipage}{.5\textwidth}
    \begin{equation*}
        \begin{split}
            \matr{A} &= \begin{pmatrix}
                0 & 1 & 1 & 1 \\
                1 & 0 & 0 & 0 \\
                1 & 0 & 0 & 0 \\
                1 & 0 & 0 & 0
            \end{pmatrix} \\
            \vspace{5mm} \\
            E &= \set{(1, 2), (1, 3), (1, 4)} \\
            \vspace{5mm} \\
            \matr{G} &= \begin{pmatrix}
                0 & 1 & 1 \\
                1 & 0 & 1 \\
                1 & 1 & 0
            \end{pmatrix}
        \end{split}
    \end{equation*}
\end{minipage}
\vspace{5mm}

Given the associated line graph, the traded value vector is,

\begin{equation}
    2 P \circ \Y_t = \begin{pmatrix}
        0.75  & -0.25 & -0.25 \\
        -0.25 & 0.75  & -0.25 \\
        -0.25 & -0.25 & 0.75
    \end{pmatrix} \Delta_t
\end{equation}

