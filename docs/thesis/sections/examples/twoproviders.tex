Consider first the case of two providers drawn below.

\vspace{5mm}
\begin{minipage}{.5\textwidth}
  \resizebox{\textwidth}{!}{\tikzstyle{var} = [
draw,circle,
minimum size=10pt]

\tikzstyle{agent} = [
draw, circle,
minimum size=10pt]

\begin{tikzpicture}[-{Latex[scale=1]}, thick]

    \node [agent] (one) {Prov. $1$};
    \node [var, dashed, below = 2cm of one] (local_one) {Market $1$};

    \node [agent, right = 5cm of one] (two) {Prov. $2$};
    \node [var, dashed, below = 2cm of two] (local_two) {Market $2$};


    \path
    (one) edge [-] node [above] {$Y^{(1, 2)}$} (two)
    (local_one) edge [] node [right] {$X(p_1)$} (one)
    (local_two) edge [] node [left] {$X(p_2)$} (two);

\end{tikzpicture}}
\end{minipage}
\begin{minipage}{.45\textwidth}
  \begin{equation*}
    \begin{split}
      \matr{A} &= \begin{pmatrix}
        0 & 1 \\
        1 & 0
      \end{pmatrix} \\
      \G &= \begin{pmatrix}
        0
      \end{pmatrix}
    \end{split}
  \end{equation*}
\end{minipage}
\vspace{5mm}

Using the bargaining solution, equation (\ref{bargaining_solution}), we obtain the trading price,

\begin{equation}
  P^{(1, 2)}_t = \frac{X_{1, t} \  p_{1, t} - X_{2, t} \  p_{2, t}}{2 Y^{(1, 2)}_t}
\end{equation}

Satisfaction of the constraint requires that, $X_{1, t} = - X_{2, t} = Y^{(1, 2)}_t$, hence,

\begin{equation} \label{two:P}
  \begin{split}
    P^{(1, 2)}_t &= \frac{Y^{(1, 2)}_t \  \left( p_{1, t} + p_{2, t} \right)}{2  Y^{(1, 2)}_t} \\
    &= \frac{p_{1, t} + p_{2, t}}{2}
  \end{split}
\end{equation}

Equation (\ref{two:P}) tells us that the influence of each node on the traded price is simply $1 / 2$. Note that in this is consistent with the influence matrix,

\begin{equation}
  (2 \I_{1} + \G)^{-1} = \left( 2 \begin{pmatrix} 1 \end{pmatrix} + \begin{pmatrix} 0 \end{pmatrix} \right)^{-1} = \frac{1}{2}
\end{equation}

Using this in the policy function of one we obtain,

\begin{equation*}
  \begin{split}
    p_{1, t+1} &= p_{1, t} + \frac{a_{1, t}}{b_{1, t}} + \frac{1 - \beta}{\beta} \frac{X_{1, t}}{b_{1, t}} + 2P^{(1, 2)} \\
    &= 2p_{1, t} + p_{2, t} + \frac{a_{1, t}}{b_{1, t}} + \frac{1 - \beta}{\beta} \frac{X_{1, t}}{b_{1, t}}
  \end{split}
\end{equation*}

and, by the same procedure, the policy function of two,

\begin{equation*}
  p_{2, t+1} = 2 p_{2, t} +  p_{1, t} + \frac{a_{2, t}}{b_{2, t}} + \frac{1 - \beta}{\beta} \frac{X_{2, t}}{b_{2, t}}
\end{equation*}

Stacking vertically we obtain,

\begin{equation}
  p_{t+1} = \begin{pmatrix}
    2  & 1 \\
    -1 & 0
  \end{pmatrix} p_t + a_t \oslash b_t + \frac{1 - \beta}{\beta} \left( X_t \oslash b_t \right)
\end{equation}

This equation determines the coevolution of the two prices. Consider the case in which there is no excess demand $X_{1, t} = X_{2, t} = 0$ and prices are constant, at levels $\bar{p}$. Then the coevolution equations yields,

\begin{equation} \label{two:const}
  \bar{p} =
\end{equation}

Hence, prices are constant and no trade happens as long as the beliefs of the two providers are the same. Note that this is true in a completely symmetric case. Starting from this, consider a positive shock in the excess demand in node two at time $\tau$ ($X_{2, \tau}$ increases due to an increase in $e_{2, \tau}$). Then, using the condition (\ref{two:const}),

\begin{equation}
  \begin{split}
    p_{2, \tau + 1} &= p_2 + \frac{1 - \beta}{\beta} \  \frac{X_{2, \tau}}{b} \\
    p_{1, \tau + 1} &= p_1
  \end{split}
\end{equation}


\begin{figure}[H]
  \begin{subfigure}{0.48\textwidth}
    \centering
    \includegraphics[width = \textwidth]{\plotpath/two/pricesupply.pdf}
    \caption{Price and supply in each node}
  \end{subfigure}
  \hfill
  \begin{subfigure}{0.48\textwidth}
    \centering
    \includegraphics[width = \textwidth]{\plotpath/two/demand.pdf}
    \caption{Excess demand, $X_{1, t}$ and $X_{2, t}$, and supply shortage, $X_{1, t} + X_{2, t}$.}
  \end{subfigure}
  \caption{A demand shock with two providers} \label{fig:two}
\end{figure}

