\documentclass[../thesis.tex]{subfiles}
\section{Conclusion}

This paper studies the effects of exogenous local electricity demand increases on the cross-border electricity market to explain price imbalances between regional markets. The determinants and behavior of this mechanism are of prime importance in an energy sector with increasing presence of prosumers.

Previous literature, both empirical and theoretical, relied on market power under rational expectations or heterogeneity of agents to explain price imbalances and contagion in the cross-border market. Nevertheless, such models fail in explaining sustained imbalances in the case of increasing integration. Here I proposed a framework that relies on a market's bargaining power, given by its relative position within a network of regional market, and showed how this, paired with myopic but rational agents, is sufficient to explain such imbalances. Among acyclic graphs, I show how bargaining power concentration on the network makes the system vulnerable to demand shocks. Furthermore, by using the associated line graph, I propose a measure of bargaining power that depends only on the network structure. Finally, I evaluate price caps and material integration as policies aimed at stabilizing the system. I find the former to be effective in reducing blackouts yet ineffective in mitigating price contagion and the latter to be effective in mitigating price contagion but detrimental to blackout occurrence.

As any model, the one presented here makes a wealth of simplifying assumption. Assessing whether these are innocuous or not is a complicated problem. In particular, by giving providers the role of ``market makers'', I abstract away a lot of the sophisticated market structures (e.g. auctions, financial instruments) and the strict regulatory requirements (e.g. reliability standards) that go into determining electricity prices. Nevertheless, introducing these would increase the model complexity, probably into intractability, but would not change the fundamentals of price dynamics. On the other hand, an assumption that limits the model's scope to acyclic graphs is that of pairwise Nash bargaining. This bargaining procedure allows for a detailed mechanism of price contagion from cross-border to local markets since it generates a unique solution that depends linearly in the agents' outside options. A useful model extension would be to consider a more extensive class of network bargaining procedures. Looking forward, the model ought to be put to the empirical test. The model can be calibrated using only observables, namely regional electricity prices, cross-border prices, and local electricity prices. Nevertheless, it requires one to think more carefully of the time step in terms of production decision, which I have not done here.

In conclusion, it is the economists' role to study ways to mitigate economic frictions as we undergo this energy transition. To fulfill its duty, this paper gives a consistent, albeit partial, framework to navigate demand shocks in the electricity markets and their repercussions.