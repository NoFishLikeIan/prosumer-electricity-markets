\documentclass[../thesis.tex]{subfiles}


\subsubsection{Complete graph}
Consider the simplest non-trivial example, namely the complete graph with three nodes (Figure \ref{tikz:complete}). The optimization of prosumer one can be written as,

\begin{equation*}
    V(X_{1, t}) = \max_{p_{1, t}, Y_t^{(1, 2)}, Y_t^{(1, 3)}} \left\{ \mathcal{L}\left( p_{1, t}, Y_t^{(1, 2)}, Y_t^{(1, 3)}, X_{1, t} \right) \right\}
\end{equation*}

\begin{figure}[!ht]
    \centering
    \tikzstyle{var} = [
draw,circle,
minimum size=10pt]

\tikzstyle{agent} = [
draw, circle,
fill=yellow,
minimum size=10pt]

\begin{tikzpicture}[-, thick]

    \node [agent] (one) {Prov. $1$};
    \node [agent, above right = 2.59807621135cm and 1.3cm of one] (two) {Prov. $2$};
    \node [agent, right = 3cm of one] (three) {Prov. $3$};


    \path
    (one) edge [] node [above left] {$Y^{(1, 2)}$} (two)
    (two) edge [] node [above right] {$Y^{(2, 3)}$} (three)
    (one) edge [] node [below] {$Y^{(1, 3)}$} (three);

\end{tikzpicture}
    \caption{Complete graph}
    \label{tikz:complete}
\end{figure}

By $\frac{\partial \mathcal{L}}{\partial Y^{(1, 2)}} = 0$, the shadow price of the bargaining procedure can be written as,

\begin{equation*}
    -\lambda_1 = P^{(1, 2)} + Y^{(1, 2)} \cdot \frac{\partial P^{(1, 2)}}{\partial Y^{(1, 2)}} + Y^{(1, 3)} \cdot \frac{\partial P^{(1, 3)}}{\partial Y^{(1, 2)}}
\end{equation*}

The optimization of prosumers two and three is equivalent. Using $Y^{(2, 1)} = - Y^{(1, 2)}$, the multipler of two becomes

\begin{equation*}
    -\lambda_2 = P^{(2, 3)} + Y^{(2, 3)} \cdot \frac{\partial P^{(2, 3)}}{\partial Y^{(2, 3)}} - Y^{(1, 2)} \cdot \frac{\partial P^{(1, 2)}}{\partial Y^{(2, 3)}},
\end{equation*}

and, using $\partial Y^{(1, 3)} = -\partial Y^{(3, 1)}$, the multiplier of three becomes

\begin{equation*}
    -\lambda_3 = -P^{(1, 3)} - Y^{(1, 3)} \cdot \frac{\partial P^{(1, 3)}}{\partial Y^{(1, 3)}} - Y^{(2, 3)} \cdot \frac{\partial P^{(2, 3)}}{\partial Y^{(1, 3)}}.
\end{equation*}

We can then use the Nash bargaining price (\ref{bargaining_solution}) to define $P$,

\begin{equation*}
    P^{(1, 2)} = \underbrace{\Delta^{(1, 2)}}_{\text{revenue difference}} - \underbrace{Y^{(1, 3)} \cdot P^{(1, 3)}}_{\text{outside option of } 1} + \underbrace{Y^{(2, 3)} \cdot P^{(2, 3)}}_{\text{outside option of } 2}  \Bigg/ 2 Y^{(1, 2)}
\end{equation*}

To get an intuition for the formula, note that if we were to cut the ties between one and three (setting $Y^{(1, 3)}$ to 0), thereby removing one's outside option, we would obtain a higher price $P^{(1, 2)}$.

By stacking the price over every edge in a vector we obtain,

\begin{equation*}
    2 \cdot \underbrace{\begin{pmatrix}
            Y^{(1, 2)} P^{(1, 2)} \\
            Y^{(1, 3)} P^{(1, 3)} \\
            Y^{(2, 3)} P^{(2, 3)}
        \end{pmatrix}}_{\Y \circ P} =
    \underbrace{\begin{pmatrix}
            \Delta^{(1, 2)} \\
            \Delta^{(1, 3)} \\
            \Delta^{(2, 3)}
        \end{pmatrix}}_{\Delta} - \underbrace{\begin{pmatrix}
            0 & 1  & -1 \\
            1 & 0  & -1 \\
            1 & -1 & 0
        \end{pmatrix}}_{\G} \underbrace{\begin{pmatrix}
            Y^{(1, 2)} P^{(1, 2)} \\
            Y^{(1, 3)} P^{(1, 3)} \\
            Y^{(2, 3)} P^{(2, 3)}
        \end{pmatrix}}_{\Y \circ P}
\end{equation*}

This can be rewritten as,

\begin{equation*}
    \begin{split}
        \left( \begin{pmatrix}
                2 & 0 & 0 \\
                0 & 2 & 0 \\
                0 & 0 & 2
            \end{pmatrix} + \begin{pmatrix}
                0 & 1  & -1 \\
                1 & 0  & -1 \\
                1 & -1 & 0
            \end{pmatrix} \right) (\Y \circ P) &= \Delta \\
        \begin{pmatrix}
            2 & 1  & -1 \\
            1 & 2  & -1 \\
            1 & -1 & 2
        \end{pmatrix} (\Y \circ P) &= \Delta \\
        \begin{pmatrix}
            Y^{(1, 2)} & 0          & 0          \\
            0          & Y^{(1, 3)} & 0          \\
            0          & 0          & Y^{(2, 3)}
        \end{pmatrix}
        P
        &= \begin{pmatrix}
            2 & 1  & -1 \\
            1 & 2  & -1 \\
            1 & -1 & 2
        \end{pmatrix}^{-1} \Delta \\
        P
        = \underbrace{\begin{pmatrix}
                1 / Y^{(1, 2)} & 0              & 0              \\
                0              & 1 / Y^{(1, 3)} & 0              \\
                0              & 0              & 1 / Y^{(2, 3)}
            \end{pmatrix}}_{\diag(\Y)^{-1}} &\begin{pmatrix}
            0.5  & -0.167 & 0.167 \\
            -0.5 & 0.833  & 0.167 \\
            -0.5 & 0.5    & 0. 5
        \end{pmatrix} \Delta \\
    \end{split}
\end{equation*}

Letting $D_{\Y} \coloneqq \frac{\der}{\der \Y}$ be the derivative with respect to the vector $\Y$, taking the derivative in the above expression yields,

\begin{equation}
    \begin{split}
        \der P &= \der \diag(\Y)^{-1} \begin{pmatrix}
            0.36  & -0.07 & 0.21  \\
            0.21  & 0.36  & -0.07 \\
            -0.07 & 0.21  & 0.36  \\
        \end{pmatrix} \Delta \\
        \der P &= -\diag(\Y)^{-1} \der \diag(\Y) \underbrace{\diag(\Y)^{-1} \begin{pmatrix}
                0.36  & -0.07 & 0.21  \\
                0.21  & 0.36  & -0.07 \\
                -0.07 & 0.21  & 0.36  \\
            \end{pmatrix} \Delta}_{P} \\
        \der P &= -\diag(\Y)^{-1} \der \diag(\Y) P \\
        \der P &= -\diag(\Y)^{-1} \diag(P) \der \Y
    \end{split}
\end{equation}

Hence we can write the jacobian of the price vector as,

\begin{equation}
    D_{\Y} P = -\begin{pmatrix}
        P^{(1, 2)} / Y^{(1, 2)} & 0                       & 0                       \\
        0                       & P^{(1, 3)} / Y^{(1, 3)} & 0                       \\
        0                       & 0                       & P^{(2, 3)} / Y^{(2, 3)}
    \end{pmatrix}
\end{equation}

Using the multiplier's definition this allows as to rewrite,

\begin{equation}
    \begin{split}
        -\lambda_1 &= 2P^{(1, 2)} \\
        -\lambda_2 &= 2P^{(2, 3)} \\
        -\lambda_3 &= -2P^{(1, 3)}
    \end{split}
\end{equation}

Finally, using the policy function (\ref{local_p}), the evolution of local prices,

\begin{equation*}
    \begin{split}
        p_{1, t+1} &= p_{1, t} + \frac{a_{1, t}}{b_{1, t}} + \left( \frac{1 - \beta}{\beta}\right) \frac{X_{1, t}}{b_{1, t}} + 2 P^{(1, 2)} \\
        p_{2, t+1} &= p_{2, t} + \frac{a_{2, t}}{b_{2, t}} + \left( \frac{1 - \beta}{\beta}\right) \frac{X_{2, t}}{b_{2, t}} + 2 P^{(2, 3)} \\
        p_{3, t+1} &= p_{3, t} + \frac{a_{3, t}}{b_{3, t}} + \left( \frac{1 - \beta}{\beta}\right) \frac{X_{3, t}}{b_{3, t}} -2 P^{(1 3)}
    \end{split}
\end{equation*}