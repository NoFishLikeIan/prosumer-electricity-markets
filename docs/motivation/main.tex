\documentclass{letter}

\signature{Andrea Titton}

\setlength\topmargin{-50pt}
\setlength\textheight{10in}

\begin{document}
\begin{letter}{}

    \opening{To whom it may concern,}

    I am an MPhil student at the Tinbergen Institute in Amsterdam, and I am applying to an EPOC research project. During the course of my studies, my interest has been on theoretical and mathematical economics, focusing on traditional macroeconomics modelling, agent-based modelling, complexity, networks, and game theory. On top of this, I have professional experience and keen interest in computer science and computational methods. Working in a commercial setting taught me the value of interpretation and feasibility of modelling, both statistical and theoretical: seeing models as a means to communicate ideas and make informed decision, rather than an end in itself.

    The EPOC project provides a great application of the my skills to a socially relevant and challenging array of issues, in line with my research interests. Project 9 in particular would allow me to to combine graph theory, dynamical systems, computational methods, and economic modelling, while exploring the literature on electricity markets. I find equilibria and self-organizing criticalities in diffused market systems fascinating and would love to invest time into figuring out the best price formation mechanism. Understanding these diffused system can have major repercussion on the energy transition and is a fundamental step towards decarbonizing the economy in an equitable and fair manner. This is an important issue and I am motivation by doing socially impactful research. My second choice, project 5, seems like a natural fit given my skill set and interests. Expectation formation is a recurrent and crucial issue throughout macroeconomics and has always been a tough nut to crack. Research published by Prof. Hommes convinced me that heterogenous agent-based models are the way forward to connect individual and aggregate expectations, and make sensible predictions on aggregate expectations formation. My contribution to this line of research can be to identify where machine learning, which I have extensively used in a commercial setting, can help stabilize dynamics in macroeconomic models, which I have studied in an academic setting.

    In conclusion, I am committed to pursue a PhD because I feel like my time should be focused on better understanding economic issues, and being part of the research community at large. The end goal is to contribute to the understanding and decision making of topics that are impacting or will impact people. Hence, pursuing my doctoral studies within the context of your research projects represents an opportunity, on one hand, to learn and collaborate with experts on such topics and, on the other hand, to move the first steps in putting my skills to good use.

    \closing{Thank you for your consideration.}

\end{letter}
\end{document}