\documentclass[american]{scrartcl}

    \newcommand{\lang}{en}

    \usepackage{babel}
    \usepackage[utf8]{inputenc} 
    \usepackage{csquotes}
    \usepackage{amsmath, amssymb}
    \usepackage{graphicx}
    \usepackage{tikz} 

    \setlength{\parindent}{0em}
    \setlength{\parskip}{0.5em}

    % Graphs
    \usetikzlibrary{positioning}
    \tikzset{main node/.style={circle, draw,minimum size=1cm,inner sep=3pt},}

    % Math commands
    \newcommand{\E}{\mathbb{E}}


    \usepackage[
        bibencoding=utf8, 
        style=apa
    ]{biblatex}

    \bibliography{../../../Desktop/bibliographies/thesis}
    
    
    \usepackage{amsmath}
    \title{
        Master thesis abstract
    }

    \author{Andrea Titton}
    
\begin{document}

\maketitle

In recent years the economy found itself in the midst of a strong and vital public push towards decarbonisation of the energy sector. In coming years, this change in the households' role in the energy market, from consumers to prosumers, will present policy makers and economists with a vast spectrum of opportunities and challenges. In particular, the transition from centralized and oligopolistic fossil fuel producers to distributed renewable energy prosumers calls for a shift in economic modelling of energy markets, both in methodology and objectives

The project's main objective is to identify the conditions under which a decentralized prosumer market for electricity displays aggregate resilience. In order to do so, the paper develops and calibrates an agent-based model of a electricity market which displays a network structure. The dynamical properties of the model are linked to the network structure, and tested empirically using the calibrated model.


\end{document}