\documentclass{letter}

\signature{Andrea Titton}

\usepackage{changepage}

\setlength\topmargin{-90pt}
\setlength\textheight{10in}

\begin{document}
\begin{letter}{}

    \begin{adjustwidth}{-30pt}{-30pt}
        \opening{To whom it may concern,}


        My keen interest for a research career in Economics is the reason I am applying to the EPOC research projects.

        Since my undergraduate studies I have always had a curiosity for quantitative economic research and a will to understand how mathematical models can allow one to describe and predict economic variables. Studying economic theory and history was, and still is, the driving force behind my desire to work as a researcher. In line with this, I focused my studies on theoretical and mathematical economics, from traditional macroeconomics modelling and game theory to agent-based modelling and graph theory. My MPhil thesis, and hopefully bridge project to a PhD, under the supervision of Prof. Cees Diks and Prof. Florian Wagener, aims at exploring the dynamics of the interaction between prosumer and oligopolistic competition electricity markets, with particular focus on agents heterogeneity among prosumers. I chose this topic because electricity markets are inherently volatile and heterogenous therefore require one to go beyond the traditional macroeconomic toolkit and use concepts and methods from network dynamics and complex systems.

        On top of this, I have professional experience in computer science and computational methods. Working in a commercial setting taught me the value of interpretation and feasibility of modelling, both statistical and theoretical: seeing models as a mean to communicate ideas and make informed decision, rather than an end in itself. This viewpoint would facilitate collaboration in a research setting. Furthermore, my expertise can support my research by smoothing the practicalities that a researcher faces when dealing with the day-to-day work, for instance coding or data manipulation.


        The EPOC project provides a great opportunity to apply my skills, in theoretical modeling and computational methods, to a socially relevant and challenging array of issues, in line with my research interests. Project 9 in particular would allow me to to combine graph theory, dynamical systems, computational methods, and economic modelling, while exploring the literature on electricity markets. I find equilibria and self-organizing criticalities in diffused market systems fascinating and would love to invest time into figuring out the best price formation mechanism. Understanding these diffused system can have major repercussion on the energy transition and is a fundamental step towards decarbonizing the economy in an equitable and fair manner. This is an important issue and I am motivated by doing socially impactful research. My second choice, project 5, seems like a natural fit given my skill set and interests. Expectation formation is a recurrent and crucial issue throughout macroeconomics and has always been a tough nut to crack. Research published by Prof. Cars Hommes convinced me that heterogenous agent-based models are the way forward to connect individual and aggregate expectations, and make sensible predictions on aggregate expectations formation. My contribution to this line of research can be to identify where machine learning, which I have extensively used in a commercial setting, can help stabilize dynamics in macroeconomic models, which I have studied in an academic setting.

        In conclusion, I am committed to pursue a PhD because I aim at better understanding economic issues, and being part of the research community at large. The end goal is to contribute to the understanding and decision making of topics that are impacting or will impact people. Hence, pursuing my doctoral studies within the context of your research projects represents an opportunity, on one hand, to learn and collaborate with experts on such topics and, on the other hand, to start putting my skills to good use.
        \closing{Thank you for your consideration.}
    \end{adjustwidth}

\end{letter}
\end{document}