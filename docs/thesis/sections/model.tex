\documentclass[../main.tex]{subfiles}
\section{Model}

The model focuses on two aspects of the cross-border electricity market. First, it models explicitly the bilateral trading procedure among electricity providers in the wholesale electricity market, which is a common mechanism of price setting, particularly in cross-border daily contracts in the EU and USA (\cite{Imran2014}). Doing this requires simplifying greatly the electricity market structure, particularly the different time scales at which operators trade, the intra-border competition, and the complex financial instruments. Nevertheless, these aspects can be approximated in my model by tweaking the network structure and these simplifications allows to concentrate on the bargaining-power induced by the graph, and the instabilities these might induce.

\begin{figure}[!ht]
    \centering
    \tikzstyle{prosumers} = [
draw,circle,
prosumer,
fill=prosumer,
text=black,
minimum size=5pt]

\tikzstyle{provider} = [
draw,circle,
minimum size=10pt]

\tikzstyle{market} = [
draw=gray, dashed, thick,
inner sep=8pt]


\begin{tikzpicture}[{Latex[scale=0.5]}-{Latex[scale=0.5]}, thick]

    % Placing providers
    \foreach \x/\y [count=\j] in {1/1, -1/1, 1/-1, -1/-1}
        % Draw provider
        {\node [provider]  (\j) at (2*\x, 2*\y) [fontscale=0.8] {\makecell[l]{Prov. \j \\ $X(p_{\j})$}};

            % Draw prosumers
            \edef\points{}
            \foreach \z/\w [count=\i] in {0/2, 1/2.1, 1.75/1.75, 2.1/1, 2/0}
                {\node [prosumers] (\x\y\i) at (\x + 1.3*\x*\z, \y + 1.5*\y*\w) [fontscale=0.6] {$x_{\i}$};
                    \path (\j) edge [] node [fontscale = 0.6] {} (\x\y\i);
                    \xdef\points{(\x\y\i) \points}
                };}

    \path
    (1) edge [-{Latex}] node [above, fontscale=0.8] {$Y^{(1, 2)}$} (2)
    (1) edge [-{Latex}] node [right, fontscale = 0.8] {$Y^{(1, 3)}$} (3)
    (1) edge [-{Latex}] node [above left, fontscale = 0.8] {$Y^{(1, 4)}$} (4)
    (3) edge [-{Latex}] node [above, fontscale = 0.8] {$Y^{(3, 4)}$} (4);

\end{tikzpicture}
    \caption{A stylized model structure}
    \label{tikz:model}
\end{figure}

\begin{figure}[!ht]
    \centering
    \tikzstyle{var} = [
draw,circle,
prosumer,
fill=prosumer,
text=black,
minimum size=10pt]

\tikzstyle{agent} = [
draw, circle,
fill=blue!30,
minimum size=10pt]

\tikzstyle{derived} = [
draw, circle, dashed,
minimum size=10pt]

\tikzstyle{time} = [
draw=gray, rectangle,
dashed,
thick,
inner sep=5pt]

\tikzstyle{market} = [
draw=gray, circle,
dashed,
thick,
inner sep=5pt]


\begin{tikzpicture}[-{Latex[scale=1]}, thick, every text node part/.style={align=center, fontscale=0.8}]

    \node [agent] (prov) {Provider};
    \node [market, left = 2cm of prov] (cb-market) {Cross-border \\ market};
    \node [agent, above right = 1cm and 5cm of prov] (prod) {Producers};
    \node [agent, below right = 1cm and 5cm of prov] (pros) {Prosumers};
    \node [market, right = 2cm of prov] (w-market) {Local \\ wholesale \\ market};

    \path
    (cb-market) edge [bend left] node [above] {$Y_t$} (prov)
    (prov) edge [dashed, bend left] node [below] {$P_t$} (cb-market)
    (prov) edge [ bend left] node [above] {$X_t$} (w-market)
    (w-market) edge [dashed, bend left] node [below] {$p_t$} (prov)
    (prod) edge [] node [above left] {$S_t$} (w-market)
    (w-market) edge [{Latex[scale=1]}-{Latex[scale=1]}] node [below left] {$e_t \  M$} (pros)
    ;

\end{tikzpicture}
    \caption{Model timing}
    \label{tikz:timing}
\end{figure}

\subsection{Prosumers}

Prosumers are endowed an electricity quantity. The endowment follows a Markov process with some persistence $\rho$, $e_t \sim \matr{\pi}_{e}$, and a low and high state $\overline{e}$ and $\underline{e}$. Prosumers have a constant electricity consumption, the rest is sold. This can be encoded by making,

\begin{equation}
    \overline{e} > 0 \text{ and } \underline{e} < 0
\end{equation}

\subsection{Producers}

Let $s$ be the electricity production and $c(s)$, $c^\prime(s) > 0$ the cost of ramping up or turning down production next period by $r$. Producers are assumed to be naive over the price they face in the market, namely, $\E_{\psi}[p_{t+1}] = \psi \cdot p_{t}$. The Bellman equation is,

\begin{equation} \label{bellman_prod}
    \begin{split}
        V(s_t, p_t) &= \max_{r_t \in [\underline{r}, \bar{r}]} \left\{ s_t \cdot p_t - c(s_t) \cdot r_t + \beta \cdot V(s_{t+1}, p_{t+1}) \right\} \\
        \text{ s.t. } s_{t+1} &= s_t + r_t \text{ with } s_t \geq 0, s_0 \\
        p_{t+1} &= \psi_t \cdot p_t \\
        \underline{r} &< 0 < \bar{r}
    \end{split}
\end{equation}

Let,

\begin{equation}
    J_{\psi}(s, p, r) = s \cdot p - c(s) \cdot r + \beta \cdot V(s + r, \psi \cdot p)
\end{equation}

such that, for a given $\psi$, Equation (\ref{bellman_prod}) can be written as,

\begin{equation}
    V(s, p) = \max_{r_t \in [\underline{r}, \bar{r}]} J_{\psi}(s, p, r).
\end{equation}

Maximization of $J_\psi$ requires (first order condition),

\begin{equation} \label{J_foc_r}
    \frac{\partial J_{\psi}}{\partial r}(s, p, r)  = -c(s) + \beta \cdot   \frac{\partial V}{\partial s}(s+r, \psi \cdot p)  = 0
\end{equation}

Assuming maximization in $r$, by the envelope theorem,

\begin{equation}
    \begin{split}
        \frac{\partial V}{\partial s}(s, p) &= p - c^\prime(s) \cdot r + \beta \cdot  \frac{\partial V}{\partial s}(s+r, \psi \cdot p)
    \end{split}
\end{equation}

Using Equation (\ref{J_foc_r}) on the first condition yields,

\begin{equation}
    \begin{split}
        &\frac{\partial V}{\partial s}(s, p) = p - c^\prime(s) \cdot r + c(s) \\
        \implies &\frac{\partial V}{\partial s}(s + r, \psi \cdot p) = \psi \cdot p - c^\prime(s + r) \cdot r +  c(s + r)
    \end{split}
\end{equation}

Using (\ref{J_foc_r}),

\begin{equation}
    c^\prime(s + r) \cdot r - c(s+r) = \psi \cdot p - \frac{c(s)}{\beta}
\end{equation}


\subsubsection{Polynomial costs}

If costs are $c(x) = \sum^d_{k=0} c_k \cdot x^k$, for some odd $d > 1$ ($d = 1 \implies r = 0$), then $c^\prime(x) = \sum^d_{k=1} k \cdot c_k \cdot x^{k-1}$.

\begin{equation}
    \begin{split}
        c^\prime(s + r) \cdot r - c(s+r) &= r \cdot \sum^d_{k=1} k \cdot c_k \cdot (s+r)^{k-1} - \sum^d_{k=0} c_k \cdot (s+r)^k \\
        &= \sum^d_{k=1} r \cdot k \cdot c_k \cdot (s+r)^{k-1} - \sum^d_{k=1} c_{k-1} \cdot (s+r)^{k-1} \\
        &= \sum^d_{k=1} (r \cdot k \cdot c_k - c_{k-1}) \cdot (s+r)^{k-1}
    \end{split}
\end{equation}

Then,

\begin{equation*}
    \begin{split}
        \frac{\partial}{\partial p} \left( c^\prime(s + r) \cdot r - c(s+r) \right) &=\frac{\partial r}{\partial p} \cdot \sum^d_{k=1} k \cdot (s + r)^{k-1} + (k-1) \cdot  (r \cdot k \cdot c_k - c_{k-1}) \cdot (s+r)^{k-2}  \\
        &= \frac{\partial r}{\partial p} \cdot \sum^d_{k=1} \left( k + (k-1) \cdot  \frac{r \cdot k \cdot c_k - c_{k-1}}{s+r}\right) \cdot (s+r)^{k-1}
    \end{split}
\end{equation*}

which implies,

\begin{equation}
    \frac{\partial r}{\partial p} = \frac{\psi}{\sum^d_{k=1} \left( k + (k-1) \cdot  \frac{r \cdot k \cdot c_k - c_{k-1}}{s+r}\right) \cdot (s+r)^{k-1}}
\end{equation}

\subsubsection{Quadratic costs}

Assume $c(x) = c\cdot x^2$, $c^\prime(x) = 2c \cdot x$,

\begin{equation}
    \begin{split}
        2c \cdot (s + r) \cdot r - c \cdot (s + r)^2 &= \psi \cdot p - \frac{c \cdot s^2}{\beta} \\
        r^2 = \frac{\psi \cdot p}{c} - \left(\frac{1 - \beta}{\beta}\right) \cdot s^2
    \end{split}
\end{equation}

Maybe,

\begin{equation}
    r(s, p; \psi) =  \sqrt{\frac{\psi \cdot p}{c} - \left(\frac{1 - \beta}{\beta}\right) \cdot s^2}
\end{equation}

hence,

\begin{equation}
    \frac{\partial}{\partial p} r =\frac{\psi}{2 \cdot r(s, p; \psi)}
\end{equation}

\subsubsection{Local demand}

The total electricity supplied by $N$ producers is,

\begin{equation}
    S_{t+1} = S_t + \sum^N_{i = 1} r(s_{i, t}, p_t; \psi_{i, t})
\end{equation}

$M$ prosumers demand $M \cdot e_t$, hence the supply is,

\begin{equation}
    X_{t+1} = M \cdot e_{t+1} - S_{t+1}
\end{equation}

which yields the evolution of demand,

\begin{equation}
    X_{t+1}  = X_t + M \cdot \left(e_{t+1} - e_t \right) -  \sum^N_{i = 1} r(s_{i, t}, p_t; \psi_{i, t})
\end{equation}

\subsection{Providers}

\subsubsection{Notation}

Grid firms operate on a graph $\mathcal{A} = (V, E)$. Firms are nodes, $i \in V$, and they can trade if they share an edge, $(i, j) \in E$. I will indicate the neighbors of a node as,

\begin{equation}
    N_{\mathcal{A}}(i) = \set{ \ j \in V: \ (i, j) \in E \ }
\end{equation}

\subsubsection{Setup}

Each period, the optimization problem of the firm is,

\begin{equation}
    \max_{p_{i, t}, Y_{i, t}} \Pi_i(p_{i, t}, \matr{Y}_{i, t})
\end{equation}

where $p_{i, t} \in \R_{+}$ and

\begin{equation}
    \begin{split}
        \matr{Y}_{i, t} &\in \R^{\abs{N_{\mathcal{A}}(i)}} \\
        \matr{Y}_{i, t} &= \begin{pmatrix}
            Y^{(i, 1)}_t \\
            Y^{(i, 2)}_t \\
            \vdots       \\
            Y^{\left(i, \abs{N_{\mathcal{A}}(i)} \right)}_t
        \end{pmatrix}
    \end{split}
\end{equation}

is a vector of traded quantities with the provider's neighbors. More precisely,

\begin{equation}
    \begin{split}
        \Pi_i(p_{i, t}, \matr{Y}_{i, t}) &= p_{i, t} \cdot X_{i, t}(p_{i, t-1}) - \sum_{j \in N_{\mathcal{A}}(i)} Y_t^{(i, j)} \cdot P_t^{(i, j)} \\
        \text{ subject to } X_{i, t}(p_{i, t-1}) &=  \sum_{j \in N_{\mathcal{A}}(i)} Y_t^{(i, j)}
    \end{split}
\end{equation}

For now assume every pairwise price on the network, $P^{(i, j)}$, is determined only by the vector of traded quantities, $\matr{Y}$ with elements $Y^{(i, j)}$. Suppressing $i$ for convenience (i.e. $Y^j \coloneqq Y^{(i, j)}$), the Bellman equation is,

\begin{equation}
    \begin{split}
        V(X_t) &= \max_{p_t, \matr{Y}_t} \left\{ p_t \cdot X_t - \sum_{j \in N_{\mathcal{A}}(i)} Y^j_t \cdot P^j(\matr{Y}_t) + \lambda_t \cdot \left( X_t - \sum_{j \in N_{\mathcal{A}}(i)} Y^j_t \right) + \beta \cdot \E \ V(X_{t+1}) \right\} \\
        X_{t+1} &= X_t + M \cdot \left(e_{t+1} - e_t \right) -  \sum^N_{i = 1} r(s_{i, t}, p_t; \psi_{i, t}))
    \end{split}
\end{equation}

\subsubsection{Euler equation}

Let,

\begin{equation*}
    \begin{split}
        L(p_t, \matr{Y}_t) &\coloneqq p_t \cdot X_t - \sum_{j \in N_{\mathcal{A}}(i)} Y^j_t \cdot P^j(\matr{Y}_t) + \lambda_t \cdot \left( X_t -  \sum_{j} Y_t^j \right) \\
        &+ \beta \cdot \E \ V\left(X_t + M \cdot \left(e_{t+1} - e_t \right) -  \sum^N_{i = 1} r_{i, t} \right) \\
        \text{where } r_{i, t} &\coloneqq r(s_{i, t}, p_{t}; \psi_{i, t}) \\
    \end{split}
\end{equation*}

such that $V(X_t) = \max_{p_t, \matr{Y}_t} L(p_t, \matr{Y}_t)$. The first order condition of $L$ requires that,

\begin{equation} \label{foc_p}
    \frac{\partial}{\partial p_t} L = X_t - \beta \cdot \E \left[V^\prime (X_{t+1}) \right] \cdot \sum^N_{i = 1}  \frac{\partial}{\partial p_t} r_{i, t} = 0
\end{equation}

Furthermore, the second first order condition requires that, for all $j$,

\begin{equation} \label{foc_Y}
    \frac{\partial}{\partial Y^j_t} L =  P^j(\matr{Y}_t) + \sum_{k \in N_{\mathcal{A}}(i)} Y^k_t \cdot \frac{\partial P^k}{\partial Y^j_t} (\matr{Y}_t) - \lambda_t = 0
\end{equation}

This implies that, $\forall j, l \in N_{\mathcal{A}}(i)$,

\begin{equation} \label{lambda_before}
    \lambda_t = P^j(\matr{Y}_t) + \sum_{k \in N_{\mathcal{A}}(i)} Y^k_t \cdot \frac{\partial P^k}{\partial Y^j_t} (\matr{Y}_t) = P^l(\matr{Y}_t) + \sum_{k \in N_{\mathcal{A}}(i)} Y^k_t \cdot \frac{\partial P^k}{\partial Y^l_t} (\matr{Y}_t)
\end{equation}

\subsubsection{Envelope}

Assuming we are in the optimum, given $\partial X_{t+1} / \partial X_t = 1$,

\begin{equation} \label{env}
    V^\prime(X_t) = p_t + \lambda_t + \beta \cdot \E \ V^\prime(X_{t+1})
\end{equation}


Let,

\begin{equation}
    \begin{split}
        R_t(p, s) &\coloneqq \sum^N_{i = 1} r_{i, t} \\
        \nabla R_t(p, s) &\coloneqq \sum^N_{i = 1} \frac{\partial r_{i, t}}{\partial p_t}
    \end{split}
\end{equation}

then combining Equations (\ref{env}) and (\ref{foc_p}), we obtain

\begin{equation}
    V^\prime(X_t) = p_t + \lambda_t + \frac{X_t}{\nabla R_t}
\end{equation}

Iterating forward,

\begin{equation}
    \begin{split}
        V^\prime(X_{t+1}) &= p_{t+1} + \lambda_{t+1} + \frac{X_{t+1}}{\nabla R_{t+1}} \\
        \E \ V^\prime(X_{t+1}) &= p_{t+1} + \E[\lambda_{t+1}] + \frac{\E[X_{t+1}]}{\nabla R_{t+1}}
    \end{split}
\end{equation}

Using (\ref{foc_p}),

\begin{equation}
    \frac{X_t}{\beta \cdot \nabla R_t} = p_{t+1} + \E[\lambda_{t+1}] + \frac{\E[X_{t+1}]}{\nabla R_{t+1}}
\end{equation}

hence we obtain the policy function,

\begin{equation}
    p_{t+1} =  \frac{1}{\beta} \cdot \frac{X_t}{\nabla R_t} - \frac{\E[X_{t+1}]}{\nabla R_{t+1}} - \E[\lambda_{t+1}]
\end{equation}

where

\begin{equation}
    \E[X_{t+1}] = X_t + M \cdot \left(\E[e_{t+1}] - e_t \right) - R_t
\end{equation}


\subsection{Bargaining model}

To solve the bargaining model we need to impose a ``direction'' of trade, namely a traded quantity $Y^{(i, j)}$ enters positively in $j$ and negatively in $i$, such that

\begin{equation}
    Y^{(i, j)} = -Y^{(j, i)}
\end{equation}

Furthermore we can rewrite the summation over the neighbors using the adjacency matrix of the graph $\mathcal{A}$, such that the payoff of provider $i$ can be written as,

\begin{equation}
    \begin{split}
        \Pi_{i, t} &= X_{i, t} \cdot p_{i, t} - \sum_{j \in N_{\mathcal{A}}(i)} Y_t^{(i, j)} \cdot P_t^{(i, j)} \\
        &= X_{i, t} \cdot p_{i, t} - \sum_{j \in V} a_{i, j} \cdot Y_t^{(i, j)} \cdot P_t^{(i, j)}
    \end{split}
\end{equation}

I assume that the bargaining does not take into account individual market expectations and that it happens after the local demand generation, such that $P$ is a function of $p$ only via $Y$. The Nash bargaining solution is such that,

\begin{equation}
    P_t^{(i, j)} = \arg \max_{P_t^{(i, j)}} \left\{\Pi_{i, t} \cdot \Pi_{j, t} \right\}.
\end{equation}


which yields, for every edge $(i, j)$ with $ a_{i, j} \neq 0$,

\begin{equation} \label{bargaining_solution}
    \begin{split}
        P_t^{(i, j)} = \frac{1}{2\cdot Y_t^{(i, j)}} \Biggl( &\underbrace{X_{i, t} \cdot p_{i, t-1} - X_{j, t} \cdot p_{j, t-1}}_{\text{revenue difference }}
        \\  + &\underbrace{\sum_{m\in N\setminus \set{i}} a_{j, m} \cdot Y_t^{(j, m)} \cdot P_t^{(j, m)}}_{\text{outside option of } j}
        \\ - & \underbrace{\sum_{m \in N\setminus \set{j}} a_{i, m} \cdot Y_t^{(i, m)} \cdot P_t^{(i, m)}}_{\text{outside option of } i} \Biggr).
    \end{split}
\end{equation}


\subsection{Line graph and bargaining solution}

In the bargaining problem we have a set of directed edges $E = \set{i \to j: i, j \in N}$. On each edge a bargaining price and a transferred quantity is determined. This defines the two vectors $P, \ Y \in \R^{\abs{E}}$. We can therefore simplify the solution by appealing to the associated line graph of our original graph, with adjacency matrix $\matr{G} \in \R^{\abs{E}\times \abs{E}}$. First, let $\Delta X_t$ in $\R^{\abs{E}}$ be the vector of revenue difference with entries,

\begin{equation}
    \Delta X^{(i, j)}_t \coloneqq X_{i, t} \cdot p_{i, t-1} - X_{j, t} \cdot p_{j, t-1}
\end{equation}

At each time $t$, equation (\ref{bargaining_solution}) can be then written as,

\begin{equation} \label{matrix_bargaining_solution}
    \begin{split}
        2(P \circ Y) &= \Delta X - \matr{G} \left( P \circ Y \right) \\
        (2\matr{I} + \matr{G}) (P \circ Y) &= \Delta X \\
        (P \circ Y) &= (2\matr{I} + \matr{G})^{-1} \Delta X \\
        P &= (2\matr{I} + \matr{G})^{-1} (\Delta X \oslash Y)
    \end{split}
\end{equation}

where $\circ$ and $\oslash$ denote the element-wise (Hadamard) product and division respectively. Using the Neumann expansion of $(2\matr{I} + \matr{G})^{-1}$,

\begin{equation}
    P = \sum^{\infty}_{k=0}\frac{(-1)^k}{2^{k+1}} \matr{G}^k (\Delta X \oslash Y)
\end{equation}

\subsection{Putting it all together}

The bargaining solution tells us that the price formation follows,

\begin{equation}
    P(\matr{Y}_t) = (2\matr{I} + \matr{G})^{-1} (\Delta X_t \oslash \matr{Y}_t)
\end{equation}

We can use this, combined with the local optimization $\lambda_t$, to obtain an explicit expression of the local optimization. First, we can get rid of the Hadamard division by letting, $\matr{D}_{Y} \coloneqq \diag(\matr{Y}_t)$ and $\matr{D}_{Y} \coloneqq \diag(\matr{\Delta X})$, such that,

\begin{equation}
    P = (2\matr{I} + \matr{G})^{-1} \matr{D}_{X} \matr{D}_{Y}^{-1} \ \iota \\
\end{equation}

\begin{equation}
    \begin{split}
        \mathbb{J}\{P\}(\matr{Y}_t) &=
    \end{split}
\end{equation}

