More generally, we can study the behavior in the two ``limiting'' acyclic graphs, namely the star graph and the path graph. First, in the star graph each provider neighbors only provider one. Assume there are $n + 1$ providers/markets as below, one central and $n$ peripheral.

\vspace{5mm}
\begin{minipage}{.5\textwidth}
  \resizebox{\textwidth}{!}{\tikzstyle{var} = [
draw,circle,
minimum size=10pt]

\tikzstyle{agent} = [
draw, circle,
minimum size=10pt]

\begin{tikzpicture}[-{Latex[scale=1]}, thick]

    \node [agent] (one) {Prov. $1$};
    \node [agent, left = 3cm of one] (two) {Prov. $2$};
    \node [agent, above = 3cm of one] (three) {Prov. $3$};
    \node [agent, right = 3cm of one] (four) {Prov. $4$};


    \path
    (one) edge [] node [above] {$Y^{(1, 2)}$} (two)
    (one) edge [] node [left] {$Y^{(1, 3)}$} (three)
    (one) edge [] node [above] {$Y^{(1, 4)}$} (four);

\end{tikzpicture}}
\end{minipage}
\begin{minipage}{.5\textwidth}
  \begin{equation*}
    \begin{split}
      \matr{A} &= \begin{pmatrix}
        0      & 1 & 1 & 1 & \ldots \\
        0      & 0 & 0 & 0 &        \\
        0      & 0 & 0 &   &        \\
        0      & 0 &   &   &        \\
        \vdots &   &   &   & \ddots
      \end{pmatrix} \\
      \vspace{5mm} \\
      E &= \set{(1, 2), (1, 3), (1, 4), \ldots, (1, n)} \\
      \vspace{5mm} \\
      \G &= \matr{\iota}_n \matr{\iota}_n^T - \I_n =  \begin{pmatrix}
        0 & 1 & 1 &        \\
        1 & 0 & 1 &        \\
        1 & 1 & 0 &        \\
          &   &   & \ddots
      \end{pmatrix}
    \end{split}
  \end{equation*}
\end{minipage}
\vspace{5mm}

Notice that the number of edges is then $\abs{E} = n$. Given the associated line graph $\G$, to solve the model we need to consider the bargaining power matrix

\begin{equation*}
  \begin{split}
    (2\I_n + \G)^{-1} &= \left( 2\I_n + \matr{\iota}_n \matr{\iota}_n^T - \I_n \right)^{-1}\\
    &= \left( \I_n + \matr{\iota}_n \matr{\iota}_n^T \right)^{-1} \\
    &= \I_n - \frac{1}{n + 1} \matr{\iota}_n \matr{\iota}_n^T.
  \end{split}
\end{equation*}

Then we can use this in the bargaining solution (\ref{matrix_bargaining_solution}) to obtain

\begin{equation*}
  \begin{split}
    P_t \circ \Y_t &= (2\I_n + \G)^{-1} \Delta_t = \left( \I_n - \frac{1}{n + 1} \matr{\iota}_n \matr{\iota}_n^T \right) \Delta_t \\
    &= \Delta_t - \frac{\sum_{(i, j)} \Delta^{(i, j)}_t}{n+1} \ \matr{\iota}_n.
  \end{split}
\end{equation*}

Now focusing on the policy function of the central node, we can leverage the fact that cross-border prices are forming always in edges $(1, j)$ and therefore all have to be equal. Denoting the edge $(1, j)$ as simply $j$ and suppressing $t$, each cross-border price is

\begin{equation*}
  P^j = \frac{n\Delta^j - \sum_{m \neq j} \Delta^m }{(n + 1)Y^j}.
\end{equation*}

Equality of prices, $P^j = P^i$, implies that, for each neighbor of $1$,

\begin{equation*}
  \frac{Y^i}{Y^j} = \frac{\Delta^i - \sum_{l \neq i} \Delta^l / n}{\Delta^j - \sum_{m \neq j} \Delta^m / n}.
\end{equation*}

Then, using the feasibility constraint $X_1 = \sum^n_{j = 2} Y^j$, we can rewrite $Y^2$ as

\begin{equation*}
  Y^2 = \sum^n_{l = 2} Y^l \Big/ \sum^n_{l = 2} \frac{Y^l}{Y^2}.
\end{equation*}

Letting the average revenue difference between one and its neighbors be

\begin{equation*}
  \underline{\Delta}_1 = X_1 p_1 - \frac{1}{n-1} \sum^n_{l = 2} X_l p_l,
\end{equation*}

we can rewrite the cross-border prices as

\begin{equation*}
  P^2 = \frac{n \Delta^2 - \sum_{m \neq 2} \Delta^m}{(n-1) Y^2} = \frac{\underline{\Delta}_1}{2 (n-1) X_1} = P^j \ \forall j.
\end{equation*}

Notice that if $n = 2$, we obtain back the solution of the two providers case. Using this in the policy function of the central node

\begin{equation} \label{policy_star}
  p_{1, t+1} = p_{1, t} + L(X_{1, t}, S_{1, t}) + \frac{\underline{\Delta}_{1, t}}{(n-1) X_{1, t}}.
\end{equation}

The second fundamental structure to consider is the path graph.

\vspace{5mm}
\begin{minipage}{.5\textwidth}
  \resizebox{\textwidth}{!}{\tikzstyle{var} = [
draw,circle,
minimum size=10pt]

\tikzstyle{agent} = [
draw, circle,
minimum size=10pt]

\begin{tikzpicture}[-, thick]

    \node [agent] (one) {Prov. $1$};
    \node [agent, below right = 2cm and 2cm of one] (two) {Prov. $2$};
    \node [agent, below right = 2cm and 2cm of two] (three) {Prov. $3$};
    \node [agent, below right = 2cm and 2cm of three] (four) {Prov. $n$};


    \path
    (one) edge [] node [below left] {$Y^{(1, 2)}$} (two)
    (two) edge [] node [below left] {$Y^{(2, 3)}$} (three)
    (three) edge [dashed] node [below left] {} (four);

\end{tikzpicture}}
\end{minipage}
\begin{minipage}{.5\textwidth}
  \begin{equation*}
    \begin{split}
      \matr{A} &= \begin{pmatrix}
        0      & 1 & 0 & 0 & \ldots \\
        0      & 0 & 1 & 0 &        \\
        0      & 0 & 0 & 1 &        \\
        0      & 0 & 0 & 0 &        \\
        \vdots &   &   &   & \ddots
      \end{pmatrix} \\
      \vspace{5mm} \\
      E &= \set{(1, 2), (2, 3), (3, 4), \ldots, (n-1, n), (n, n+1)} \\
      \vspace{5mm} \\
      \G &= \begin{pmatrix}
        0      & -1 & 0  & 0  & \ldots \\
        -1     & 0  & -1 & 0  &        \\
        0      & -1 & 0  & -1 &        \\
        0      & 0  & -1 & 0  &        \\
        \vdots &    &    &    & \ddots
      \end{pmatrix}
    \end{split}
  \end{equation*}
\end{minipage}
\vspace{5mm}

Letting $\matr{L}_n$ be the lower triangular matrix with unity entries, we can rewrite the bargaining influence matrix (see \ref{a:linegraphinfluence}) as

\begin{equation}
  (\matr{2I_n + \G})^{-1} = \matr{L}_n  \left(\I - \frac{1}{n+1} \matr{\iota_n \iota_n}^T\right) \matr{L}_n^T.
\end{equation}

Then

\begin{equation*}
  P_t \circ \Y_t = \matr{L}_n  \left(\I - \frac{1}{n+1} \matr{\iota_n \iota_n}^T\right) \matr{L}_n^T  \Delta_t.
\end{equation*}

Now focusing once again on the central node $m = \frac{n+1}{2}$, assuming $n+1$ is odd, we obtain (see \ref{a:pol_path}) the ratio of exchanged quantities

\begin{equation}
  Y_r(\Delta) \coloneqq \frac{Y^{(m, m+1)}}{Y^{(m-1, m)}}
  = \frac{
    \overbrace{\frac{1}{2} \sum_{j < m} j \Delta^{(j, j+1)} + \frac{n+1}{2}\sum_{j \geq m} \left(1 - \frac{j}{n+1}\right) \Delta^{j, j+1}}^{\text{Revenue differences through the path weighted by distance to } m}
  }{
    \underbrace{\left(\frac{1}{2} + \frac{1}{n+1}\right) \sum_{j < m-1} j \Delta^{j, j+1}+ \frac{n-1}{2} \sum_{j \geq m-1} \left(1 - \frac{j}{n+1}\right) \Delta^{j, j+1}}_{\text{Revenue differences through the path weighted by distance to } m-1}
  }
\end{equation}

and, using price equality, we get the following cross-border price

\begin{equation} \label{policy_path}
  \begin{split}
    2 P(X_{m, t}, p_{m, t}, \Delta_t) &= \left( \frac{1 + 1 / Y_r(\Delta_t)}{X_m} \right) \times  \\
    &\times \left[ \overbrace{\sum_{j < m} j \Delta^{(j, j + 1)}}^{\text{distance of nodes before } m} +  \underbrace{\sum_{j \geq m}\left(n + 1 - j\right) \Delta^{(j, j + 1)}}_{\text{distance of nodes after } m} \right].
  \end{split}
\end{equation}

The star and path graph are the two extreme cases in the class of acyclic graphs, since all other acyclic graphs can be constructed as a combination of the two. This can be seen by comparing the two bargaining power matrices (Table \ref{table:influence}). In particular, the influence of an edge on each other is proportional to the distance in the associated line graph, $L(\mathcal{A})$. Given that the associated line graph of the star graph is the complete graph, each edge has distance one to each other (i.e. node one), hence the influence is $1 / (n+1)$. On the other hand, the associated line graph of the path graph is also a path graph. Hence the influence depends both on the position of the two edges in $L(\mathcal{A})$ and their relative distance.

\begin{center}
  \begin{table}[H]
    \centering
    \renewcommand{\arraystretch}{1.5}

\vspace{5mm}
\begin{tabular}{c || c | c }
                                                      & \textit{Star}                                                & \textit{Path}                                                             \\
  \midrule
  \textit{Redistribution} $(2\I + \G)^{-1}$           & $\I_n \left( \I_n - \frac{\iota \iota^T}{n+1} \right)\I_n^T$ & $\matr{L}_n \left( \I_n - \frac{\iota \iota^T}{n+1} \right) \matr{L}_n^T$ \\
  \midrule
  \textit{Individual effect} $(2\I + \G)^{-1}_{i, j}$ & $\mathbbm{1}\{i = j\} - \frac{1}{n+1}$                       & $\min\{i, j\} - \frac{i \cdot j}{n+1}$
\end{tabular}
\vspace{5mm}

    \caption{Influence matrix in the star and path graphs for $n > 3$.}
    \label{table:influence}
  \end{table}
\end{center}


Furthermore, by plotting the policy function for the central provider of the star graph, provider one, (\ref{policy_star}) and the path graph, provider $(n+1) / 2$, (\ref{policy_path}) (Figure \ref{fig:p}), one can notice how the bargaining power within the graph plays a crucial role. On the one hand, the provider in the central role of the star graph reacts more strongly to changes in its own excess demand. On the other hand, the central provider of the path graph has a weaker reaction to shocks given its lower bargaining power.

\begin{center}
  \begin{figure}[H]
    \begin{subfigure}{0.475\textwidth}
      \centering
      \includegraphics[width=\textwidth]{\plotpath/pricing.pdf}
      \caption{Star graph}
    \end{subfigure}
    \hfill
    \begin{subfigure}{0.475\textwidth}
      \centering
      \includegraphics[width=\textwidth]{\plotpath/pricingpath.pdf}
      \caption{Path graph}
    \end{subfigure}
    \caption{Contour map of the function $p_{t+1}(p_t, X_t; S_t)$ where warmer colors correspond to higher prices}
    \label{fig:p}
  \end{figure}
\end{center}

To understand how price hikes propagate through the network, as before, I simulate a shock to a regional demand from a situation of steady state. Consider a positive shock in the demand of the central market in the star graph. Figure \ref{fig:transshockcen_star} shows the price and supply evolution around the shock (shaded area) in the central node and a peripheral node. The shock tilts the system out of equilibrium and induces strong price volatility. This price volatility arises because an increase in the price of the central node, induces an increase in the price of all peripheral nodes, which in turn strongly pushes up supply across the networks. This puts downward pressure on all prices. Because of this, as opposed to the two node case, the price hike not only is much more pronounced but it is also sustained for longer. On the other hand, after the equivalent shock in the path graph the central provider does not increase prices as much, due to its lower bargaining power, and the price hike is less pronounced throughout the graph (Figure \ref{fig:transshockcen_path}).

\newpage
\begin{figure}[H]
  \centering
  \includegraphics[width = \textwidth, height = 0.4 \textheight]{\plotpath/central/star/pricesupply.pdf}
  \caption{Price and supply evolution after a transient shock in the central node (1) in a star graph} \label{fig:transshockcen_star}
\end{figure}

\begin{figure}[H]
  \centering
  \includegraphics[width = \textwidth, height = 0.4 \textheight]{\plotpath/central/path/pricesupply.pdf}
  \caption{Price and supply evolution after a transient shock in the central node of the path graph} \label{fig:transshockcen_path}
\end{figure}
\newpage

Figure \ref{fig:demandshock} shows the excess demand associated with this shock in the two graphs. The simulation once again, delivers both a trivial and a more surprising insight. Trivially, the shock in excess demand causes in both graphs a sudden shrinkage of excess demand, which is less pronounced in the case of the path graph due to a smoother price transmission and a less concentrated bargaining power. Surprisingly, the shock brings the system into a long period of oscillating production which leads to a ``blackout'' further down the line. This phenomenon arises due to the myopic view that providers have of the pricing mechanism process. In the star graph, the price hike propagates throughout the network immediately, since all nodes are linked to the central node. This causes production to increase in all peripheral nodes, which in turn generates a excess electricity supply. This overshooting creates strong downward pressure on prices, via cross-border prices and expectations of providers, which keeps supply artificially low for a sustained time period. Finally, the network structure allows for the price increase to be much more persistent. The self-fulfilling behavior persists until electricity production is high enough to push prices back down to their equilibrium levels. Importantly, a return to steady state prices happens only if the excess demand is once again satisfied throughout the network.

\begin{center}
  \begin{figure}[H]
    \begin{subfigure}{0.475\textwidth}
      \centering
      \includegraphics[width = \textwidth]{\plotpath/central/star/demand.pdf}
      \caption{In a star graph} \label{fig:demandshock_star}
    \end{subfigure} \hfill
    \begin{subfigure}{0.475\textwidth}
      \centering
      \includegraphics[width = \textwidth]{\plotpath/central/path/demand.pdf}
      \caption{In a path graph} \label{fig:demandshock_path}
    \end{subfigure}
    \caption{Excess demand $X_{i, t}$ after a transient shock in the central node} \label{fig:demandshock}
  \end{figure}
\end{center}

