\documentclass[../main.tex]{subfiles}

In recent years the economy found itself in the midst of a strong and vital public push towards decarbonisation of the energy sector. One aspect of such process is the increased adoption by households of technology to produce electricity independently, like solar panels, batteries, and ``smart'' meters (\cite{Parag2016}).

In coming years, this change in the households' role in the energy market, from consumers to prosumers, will present policy makers and economists with a vast spectrum of opportunities and challenges. In particular, the transition from centralized and oligopolistic fossil fuel producers to distributed renewable energy prosumers calls for a shift in economic modelling of energy markets, both in methodology and objectives. A shift is needed in methodology because the network structure and heterogeneity of producers add fundamental non-linearities in the market clearing mechanism. With regards to objectives, questions of efficiency and resilience in models with complex networks cannot be addressed through a traditional homogenous policy, and therefore require understanding of the heterogenous dynamics within the market structure.

The project's main objective will be to identify the conditions under which a decentralized prosumer market for electricity displays aggregate resilience.

The research question will be dealt with on three dependent levels. First, the resilience properties will be investigated within a theoretical framework. Second, based on the theoretical model, an empirical analysis will be formulated. Third, the empirical result will define suitable policy implications.