Consider first the simplest case of two providers drawn below.

\vspace{5mm}
\begin{minipage}{.5\textwidth}
  \resizebox{\textwidth}{!}{\tikzstyle{var} = [
draw,circle,
minimum size=10pt]

\tikzstyle{agent} = [
draw, circle,
minimum size=10pt]

\begin{tikzpicture}[-{Latex[scale=1]}, thick]

    \node [agent] (one) {Prov. $1$};
    \node [var, dashed, below = 2cm of one] (local_one) {Market $1$};

    \node [agent, right = 5cm of one] (two) {Prov. $2$};
    \node [var, dashed, below = 2cm of two] (local_two) {Market $2$};


    \path
    (one) edge [-] node [above] {$Y^{(1, 2)}$} (two)
    (local_one) edge [] node [right] {$X(p_1)$} (one)
    (local_two) edge [] node [left] {$X(p_2)$} (two);

\end{tikzpicture}}
\end{minipage}
\begin{minipage}{.45\textwidth}
  \begin{equation*}
    \begin{split}
      \matr{A} &= \begin{pmatrix}
        0 & 1 \\
        1 & 0
      \end{pmatrix} \\
      \G &= \begin{pmatrix}
        0
      \end{pmatrix}
    \end{split}
  \end{equation*}
\end{minipage}
\vspace{5mm}

Using the bargaining solution, equation (\ref{bargaining_solution}), we obtain the trading price,

\begin{equation}
  P^{(1, 2)}_t = \frac{X_{1, t} \  p_{1, t} - X_{2, t} \  p_{2, t}}{2 Y^{(1, 2)}_t}
\end{equation}

Satisfaction of the constraint requires that, $X_{1, t} = - X_{2, t} = Y^{(1, 2)}_t$, hence,

\begin{equation} \label{two:P}
  \begin{split}
    P^{(1, 2)}_t &= \frac{Y^{(1, 2)}_t \  \left( p_{1, t} + p_{2, t} \right)}{2  Y^{(1, 2)}_t} \\
    &= \frac{p_{1, t} + p_{2, t}}{2}
  \end{split}
\end{equation}

Equation (\ref{two:P}) tells us that the influence of each node on the traded price is simply $1 / 2$. Note that in this is consistent with the influence matrix,

\begin{equation}
  (2 \I_{1} + \G)^{-1} = \left( 2 \begin{pmatrix} 1 \end{pmatrix} + \begin{pmatrix} 0 \end{pmatrix} \right)^{-1} = \frac{1}{2}
\end{equation}

Using this in the provider's policy function,

\begin{equation*}
  \begin{split}
    p_{1, t+1} &= p_{1, t} + \frac{1-\beta}{\beta \ \gamma_{1, t}} X_{1, t} + \frac{\alpha_{1, t} + \eta_{1, t} S_{1, t}}{\gamma_{1, t}} + 2P^{(1, 2)} \\
    &= 2p_{1, t} + p_{2, t} + \frac{1-\beta}{\beta \ \gamma_{1, t}} X_{1, t} + \frac{\alpha_{1, t} + \eta_{1, t} S_{1, t}}{\gamma_{1, t}}
  \end{split}
\end{equation*}

By symmetry, provider two has the same policy function. Letting $p_t \coloneqq \begin{pmatrix}
    p_{1, t} \\ p_{2, t}
  \end{pmatrix}$ and doing so with all other variables, we can stack vertically and obtain,

\begin{equation}
  p_{t+1} = \begin{pmatrix}
    2 & 1 \\
    1 & 2
  \end{pmatrix} p_t + \left( \frac{1-\beta}{\beta} X_t + \alpha_t + \eta_t \circ S_t \right) \oslash \gamma_t
\end{equation}

This equation determines the coevolution of the two prices. To understand how price hikes propagate in the network I simulate a demand shock in the steady state model (see \ref{a:initsim} for mode details). Figure \ref{fig:two} shows the time series evolution of prices and electricity supply in the two nodes. At time $\tau \in [250, 260]$ there is an excess demand shock in node two ($e_{2, t = \tau} = 2 e_{2, t \neq \tau}$), shaded in the plot. As expected there is a ``first order'' effect of the demand shock on prices in the second regional market and then the price hike is transmitted to the first node. More interestingly, the shock pushes the model to a regime of higher price volatility and there is a second price hike wave, due to a feedback loop and excess electricity production. (\textbf{TODO: expand on this mechanism})


\begin{figure}[H]
  \centering
  \includegraphics[width = \textwidth]{\plotpath/two/pricesupply.pdf}
  \caption{Price and supply in each node, with a demand shock with two providers} \label{fig:two}
\end{figure}

Another interesting variable to track during the shock is the excess demand in each node $X_{i, t}$. In the current simulation the ``perceived'' excess demand has to be fulfilled, namely $\sum_i \B_{i, t-1}\left[X_{i, t}\right] = 0$, since its stems from the optimization problem of the provider. Nevertheless, the realized excess demand might be positive, $\sum_i X_{i, t} > 0$. In Figure \ref{fig:twodemand} I plot the excess demand in the previous simulation. The initial demand shock is not enough to put the system into a ``blackout''. But the feedback loop and increased price volatility causes a demand shock further down the line.

\begin{figure}[H]
  \centering
  \includegraphics[width = \textwidth]{\plotpath/two/demand.pdf}
  \caption{Excess demand} \label{fig:twodemand}
\end{figure}