\documentclass[../main.tex]{subfiles}
\section{Model}

The model focuses on two aspects of the cross-border electricity market. First, it models explicitly the bilateral trading procedure among electricity providers in the wholesale electricity market, which is a common mechanism of price setting, particularly in cross-border daily contracts in the EU and USA (\cite{Imran2014}). Doing this requires simplifying greatly the electricity market structure, particularly the different time scales at which operators trade, the intra-border competition, and the complex financial instruments. Nevertheless, these aspects can be approximated in my model by tweaking the network structure and these simplifications allows to concentrate on the bargaining-power induced by the graph, and the instabilities these might induce.

\begin{figure}[!ht]
    \centering
    \tikzstyle{var} = [
draw,circle,
prosumer,
fill=prosumer,
text=black,
minimum size=10pt]

\tikzstyle{agent} = [
draw, circle,
fill=blue!30,
minimum size=10pt]

\tikzstyle{onnode} = [
    draw, circle, fill=white
]

\tikzstyle{time} = [
draw=gray, rectangle,
dashed,
thick,
inner sep=5pt]

\tikzstyle{market} = [
draw=gray, circle,
dashed,
thick,
inner sep=5pt]

\begin{tikzpicture}[-{Latex}, thick, every text node part/.style={align=center, fontscale=0.8}]

    % Placing providers
    \foreach \x/\y [count=\j] in {1/1, -1/1, 1/-1, -1/-1}
        % Draw provider
        {
            \node [agent]  (\j) at (2*\x, 2*\y) [] {Provider \j};
            \node [market] (\j-mkt) at (5*\x, 5*\y) [] {Wholesale \\ market \j};
            \path 
                (\j-mkt) edge [bend right] node [onnode] {$X_{\j}$} (\j)
                (\j) edge [dashed, bend right] node [onnode, dashed] {$p_{\j}$} (\j-mkt)
            ;
        }

    \path
    (1) edge [] node [above] {$Y^{(1, 2)}$} (2)
    (1) edge [] node [right] {$Y^{(1, 3)}$} (3)
    (3) edge [] node [above] {$Y^{(3, 4)}$} (4);

\end{tikzpicture}
    \caption{A stylized model structure}
    \label{tikz:model}
\end{figure}

\subsection{Prosumer}

\subsubsection{Problem}

One instance of the model is drawn in the diagram in Figure \ref{tikz:model}. At each time step, each provider sets a \textit{local price} ($p$) at which they trade electricity with local market prosumers. Each prosumer $i$, given the endowment ($e_t$) in the local market, its cash-in-hand ($m_{i, t}$), and the current forecasting rule ($\psi_{i, t}$), consumes some electricity. The electricity consumption is given by the endowments and the demand bought from the provider, $e_t + x_{i, t}(p)$, and gains utility $u(e + x)$. If $x_{i, t}(p) < 0$, then the consumer is supplying electricity to the provider from its endowment. The endowment $e_t$ is modeled as a Markov process with a high state and a low state and transition matrix,

\begin{equation*}
    \matr{\pi}_e = \rho \cdot \matr{I}_2 + (1 - \rho)\cdot  \matr{J}_2
\end{equation*}

Suppressing $i$ for convenience, we can write the dynamic optimization problem of the prosumer as,

\begin{equation} \label{eq:pros_problem}
    \begin{split}
        V(e_t) &= \sup_{x_t \in \R} \left\{u(x_t + e_t) + \beta \cdot \E_t V( e_{t+1} ) \right\} \\
        \textit{s.t. } m_{t+1} &= m_{t} - p_{t} \cdot x_{t}\\
        m_t  &\geq 0 \textit{ given } m_0
    \end{split}
\end{equation}


Prosumers need to forecast two variables in order to make a consumption decision: the endowment $e_{t+1}$ and the price in the local market $p_{t+1}$. Prosumers are assumed to forecast the former perfectly by knowing $\matr{\pi}_e$. The latter is forecasted using a linear forecasting rule $\E_t[p_{t+1}] = \psi_t\cdot p_t + c$, where $c$ is a constant. The linear rule is selected from two possible options, $\psi_t \in \Psi = \set{0.99, 1.01}$ (\cite{Hommes2013}), based on which strategy yielded the best results in the past. Namely,

\begin{equation} % FIXME: Find this eq on Hommes
    P\left(\psi_t \neq \psi_{t-1}\right) = exp(\eta \cdot U).
\end{equation}

\subsubsection{Solution}

At time $t$, given the decision rule $\psi_t$, the optimization problem \ref{eq:pros_problem}, requires that,

\begin{equation} \label{eq:pros_euler}
    u^\prime\left( e_t + \frac{m_t - m_{t+1}}{p_t} \right) = \beta \cdot \E_t \left[ u^\prime\left(e_{t+1} + \frac{m_{t+1} - m_{t+2}}{ \psi_h \cdot p_t + c} \right)  \right].
\end{equation}


The endogenous grid method can then be used to numerically find a policy function

\begin{equation}
    m^\prime(m_t \ \vert \ \psi_t, p_t, e_t).
\end{equation}

such that each period demand for each prosumer is,

\begin{equation}
    x_{i, t}(p_t \ \vert \ \psi_{i, t}, m_{i, t}, e_t) = \frac{m_{i, t} - m^\prime(m_{i, t} \ \vert \ \psi_{i, t}, e_t, p_t)}{p_t}
\end{equation}

The solution of the local problem yields the electricity demand in a local market,

\begin{equation}
    X_t (p \ \vert \ \psi_t, m_t, e_t) = \sum^n_{i = 1} x_{i, t}(p_t \ \vert \ \psi_{i, t}, m_{i, t}, e_t)
\end{equation}

\subsection{Provider}