
\subsection{Three firms}

Assume there are three grid firms, \hspace{2em}

\vspace{0.5cm}
\begin{minipage}{0.6\textwidth}
  \resizebox{\textwidth}{!}{
    \begin{tikzpicture}[{Latex[scale=1.25]}-{Latex[scale=1.25]}, thick]
      % Grids
      \node[main node] (1) {$1$};
      \node[main node] [right = 2cm of 1] (2) {$2$};
      \node[main node] [right = 2cm of 2] (3) {$3$};
      % Paths
      \path[draw,thick]
      (1) edge node [above] {$(1, 2)$} (2)
      (2) edge node [above] {$(2, 3)$} (3);
    \end{tikzpicture}
  }
\end{minipage} \hfill
\begin{minipage}{0.35\textwidth}
  \begin{equation*}
    \matr{A} = \begin{pmatrix}
      0   & 1  & 0 \\
      - 1 & 0  & 1 \\
      0   & -1 & 0
    \end{pmatrix}
  \end{equation*}
\end{minipage}
\vspace{0.5cm}


and the associated line graph is,

\vspace{0.5cm}
\begin{minipage}{0.6\textwidth}
  \resizebox{0.7\textwidth}{!}{
    \begin{tikzpicture}[{Latex[scale=1.25]}-{Latex[scale=1.25]}, thick]
      % Grids
      \node[main node] (12) {$(1, 2)$};
      \node[main node] [right = 2cm of 1] (23) {$(2, 3)$};
      % Paths
      \path[draw,thick]
      (12) edge node [above] {$2$} (23);
    \end{tikzpicture}
  }
\end{minipage} \hfill
\begin{minipage}{0.35\textwidth}
  \begin{equation*}
    \matr{G} = \begin{pmatrix}
      0  & 1 \\
      -1 & 0
    \end{pmatrix}
  \end{equation*}
\end{minipage}
\vspace{0.5cm}


The Nash bargaining solution is then,

\begin{equation*}
  \begin{split}
    \begin{pmatrix}
      P^{(1, 2)} \\
      P^{(2, 3)}
    \end{pmatrix} &= (2\matr{I} + \matr{G})^{-1} \begin{pmatrix}
      (\Delta X / Y)^{(1, 2)} \\
      (\Delta X / Y)^{(2, 3)}
    \end{pmatrix} \\
    &= \begin{pmatrix}
      2  & 1 \\
      -1 & 2
    \end{pmatrix}^{-1} (\Delta X \oslash Y)\\
    &= \frac{1}{2 \cdot 2 - 1 \cdot (-1)} \begin{pmatrix}
      2 & - 1 \\
      1 & 2
    \end{pmatrix} (\Delta X \oslash Y) \\
    &= \frac{1}{5} \begin{pmatrix}
      2 & - 1 \\
      1 & 2
    \end{pmatrix} (\Delta X \oslash Y)
  \end{split}
\end{equation*}

In Appendix \ref{a:three_firms} I derive the matrix $(2 \matr{I} + \matr{G})^{-1}$ as a Neumann series.

Using then Equation (\ref{foc_grid}),

\begin{equation}
  \begin{split}
    P + \partial P \circ Y &= \left[ (2 \matr{I} + \matr{G})^{-1} - \diag(2 \matr{I} + \matr{G})^{-1}  \right] (\Delta X \oslash Y) \\
    &= \frac{1}{5} \left[ \begin{pmatrix}
        2 & - 1 \\
        1 & 2
      \end{pmatrix} -  \begin{pmatrix}
        2 & 0 \\
        0 & 2
      \end{pmatrix}\right] (\Delta X \oslash Y) \\
    &= -\frac{1}{5} \matr{G} (\Delta X \oslash Y)
  \end{split}
\end{equation}

Leveraging the fact that 3 and 1 have only one neighbor to trade with, $N(3) = N(1) = \set{2}$, in equilibrium their demand needs to equal the trade with 2. Namely,

\begin{equation}
  X_3 = Y^{2, 3} \text{ and } X_1 = Y^{1, 2}
\end{equation}

hence,

\begin{equation}
  \begin{split}
    \Delta X \oslash Y &= \begin{pmatrix}
      \frac{X_1 \cdot p_1 - X_2 \cdot p_2}{X_1} &
      \frac{X_2 \cdot p_1 - X_3 \cdot p_2}{X_3}
    \end{pmatrix}^{T} \\
    &= \begin{pmatrix}
      p_1 - \frac{X_2}{X_1} \cdot p_2 &
      \frac{X_2}{X_3} \cdot p_2 - p_3
    \end{pmatrix}^{T}
  \end{split}
\end{equation}

Using this we can rewrite

\begin{equation}
  P + \partial P \circ Y = - \frac{1}{5} \matr{G} \begin{pmatrix}
    p_1 - \frac{X_2}{X_1} \cdot p_2 \\
    \frac{X_2}{X_3} \cdot p_2 - p_3
  \end{pmatrix} = \frac{1}{5} \begin{pmatrix}
    p_3 - \frac{X_2}{X_3} \cdot p_2 \\
    p_1 - \frac{X_2}{X_1} \cdot p_2
  \end{pmatrix}
\end{equation}

The firms' three first order condition now be used to find $X_1, X_2, X_3$,

\begin{alignat*}{3}
  p_2 + \frac{X_2}{\partial X_2 / p_2} &  & = \frac{1}{5} \left( p_3 - \frac{X_2}{X_3} \cdot p_2  \right) &  & = p_1 + \frac{X_1}{\partial X_1 / p_1} \\
  p_2 + \frac{X_2}{\partial X_2 / p_2} &  & = \frac{1}{5} \left( p_1 - \frac{X_2}{X_1} \cdot p_2  \right) &  & = p_3 + \frac{X_3}{\partial X_3 / p_3}
\end{alignat*}

In order to simplify, let,

\begin{equation}
  F_i := p_i +  \frac{X_i}{\partial X_i / p_i}
\end{equation}

The condition implies,

\begin{equation}
  F_1 = F_2 = F_3 = \frac{p_1}{5} + \left(- \frac{X_2 \cdot p_2}{5} \right) \cdot \frac{1}{X_1} = \frac{p_3}{5} + \left(- \frac{X_2 \cdot p_2}{5} \right) \cdot \frac{1}{X_3}
\end{equation}

We can use this equation to eliminate any dependence on $p_2$,

\begin{equation}
  \begin{split}
    \frac{p_1}{5} + \left(- \frac{X_2 \cdot p_2}{5} \right) \cdot \frac{1}{X_1} &= F_3 \\
    - \frac{X_2 \cdot p_2}{5} &= \left( F_3 - \frac{p_1}{5} \right) \cdot X_1
  \end{split}
\end{equation}

which implies

\begin{equation}
  \begin{split}
    \frac{p_3}{5} + \left(- \frac{X_2 \cdot p_2}{5} \right) \cdot \frac{1}{X_3} &= F_1 \\
    \frac{p_3}{5} + \left( F_3 - \frac{p_1}{5} \right) \cdot \frac{X_1}{X_3} &= F_1 \\
    \frac{1}{5} \cdot (p_3 - p_1) + \frac{X_1}{X_3} \cdot F_3 - F_1 &= 0
  \end{split}
\end{equation}

In equilibrium, $F_3 = F_1$, hence, writing the explicit dependence on $p_i$,

\begin{equation}
  \frac{1}{5} \cdot (p_3 - p_1) + \left( \frac{X_1(p_1)}{X_3(p_3)} - 1 \right) \cdot F_1(p_1)= 0
\end{equation}

\subsection{Bottleneck}


\vspace{0.5cm}

\begin{center}
  \resizebox{0.8\textwidth}{!}{
    \begin{tikzpicture}[{Latex[scale=1.25]}-{Latex[scale=1.25]}, thick]
      % Grids
      \node[main node] (23) {$(2, 3)$};
      \node[main node] [below left = 1cm and 2cm of 23] (02) {$(0, 2)$};
      \node[main node] [above left = 1cm and 2cm of 23] (12) {$(1, 2)$};
      \node[main node] [right = 2cm of 23] (34) {$(3, 4)$};
      \node[main node] [below right = 1cm and 2cm of 34] (45) {$(4, 5)$};
      \node[main node] [above right = 1cm and 2cm of 34] (46) {$(4, 6)$};

      % Paths
      \path[draw,thick]
      (02) edge node [above] {$2$} (23)
      (12) edge node [above] {$2$} (23)
      (23) edge node [above] {$3$} (34)
      (45) edge node [above] {$4$} (34)
      (46) edge node [above] {$4$} (34);

      % Weights
      \path[draw, thick, dashed, -{Latex[scale=1.25]}]
      (02) edge [color=blue] node [left] {$.15$} (12)
      (02) edge [bend left, color=blue] node [above left] {$.075$} (23)
      (02) edge [bend right, color=red] node [below right] {$.55$} (34)
      (02) edge [color=blue] node [below right] {$.025$} (45);
    \end{tikzpicture}
  }
\end{center}
\begin{equation*}
  (\matr{2I - G})^{-1} = \begin{pmatrix}
    0.425  & -0.15 & -0.075 & 0.05 & -0.025 & -0.025 \\
    0.15   & 0.3   & 0.15   & -0.1 & 0.05   & 0.05   \\
    -0.075 & -0.15 & 0.425  & 0.05 & -0.025 & -0.025 \\
    0.05   & 0.1   & 0.05   & 0.3  & -0.15  & -0.15  \\
    0.025  & 0.05  & 0.025  & 0.15 & 0.425  & -0.075 \\
    0.025  & 0.05  & 0.025  & 0.15 & -0.075 & 0.425  \\
  \end{pmatrix}
\end{equation*}
\vspace{0.5cm}




\subsection{Alternative path}

Consider now the more complex example,

\vspace{0.5cm}
\resizebox{\textwidth}{!}{
  \begin{tikzpicture}[{Latex[scale=1.25]}-{Latex[scale=1.25]}, thick]
    % Grids
    \node[main node] (1) {$1$};
    \node[main node] [right = 2cm of 1] (2) {$2$};
    \node[main node] [right = 2cm of 2] (3) {$3$};
    \node[main node] [below = 2cm of 2] (4) {$4$};
    \node[main node] [right = 6cm of 2] (6) {$6$};
    \node[main node] [below = 2cm of 6] (5) {$5$};
    \node[main node] [right = 2cm of 6] (7) {$7$};
    % Paths
    \path[draw,thick]
    (1) edge node [above] {$(1, 2)$} (2)
    (2) edge node [above] {$(2, 3)$} (3)
    (3) edge node [above] {$(3, 6)$} (6)
    (2) edge node [left] {$(2, 4)$} (4)
    (4) edge node [below] {$(4, 5)$} (5)
    (6) edge node [right] {$(6, 5)$} (5)
    (6) edge node [above] {$(6, 7)$} (7);
  \end{tikzpicture}
}

\begin{equation*}
  \matr{A} = \begin{pmatrix}
    0  & 1  & 0  & 0  & 0  & 0  & 0 \\
    -1 & 0  & 1  & 1  & 0  & 0  & 0 \\
    0  & -1 & 0  & 0  & 0  & 1  & 0 \\
    0  & -1 & 0  & 0  & 1  & 0  & 0 \\
    0  & 0  & 0  & -1 & 0  & 1  & 0 \\
    0  & 0  & -1 & 0  & -1 & 0  & 1 \\
    0  & 0  & 0  & 0  & 0  & -1 & 0
  \end{pmatrix}
\end{equation*}

The associated link graph is,

\vspace{0.5cm}
\begin{minipage}{0.6\textwidth}
  \resizebox{\textwidth}{!}{
    \begin{tikzpicture}[{Latex[scale=1.25]}-{Latex[scale=1.25]}, thick]
      % Grids
      \node[main node] (12) {$(1, 2)$};
      \node[main node] [right = 2cm of 12] (23) {$(2, 3)$};
      \node[main node] [right = 1.5cm of 23] (36) {$(3, 6)$};
      \node[main node] [below = 2cm of 12] (24) {$(2, 4)$};
      \node[main node] [right = 2cm of 24] (45) {$(4, 5)$};
      \node[main node] [right = 1.5cm of 45] (56) {$(5, 6)$};
      \node[main node] [above right = 0.75cm and 1cm of 56] (67) {$(6, 7)$};
      % Paths
      \path[draw,thick]
      (12) edge node [above] {$2$} (23)
      (12) edge node [above] {$2$} (24)
      (45) edge node [above] {$4$} (24)
      (56) edge node [above] {$5$} (45)
      (23) edge node [above] {$3$} (36)
      (36) edge node [above right] {$6$} (67)
      (56) edge node [below right] {$6$} (67);
    \end{tikzpicture}
  }
\end{minipage} \hfill
\begin{minipage}{0.35\textwidth}

  \begin{equation*}
    \matr{G}(\matr{A}) = \begin{pmatrix}
      0  & 1  & 1  & 0  & 0  & 0 & 0  \\
      -1 & 0  & 0  & 1  & 0  & 0 & 0  \\
      -1 & 0  & 0  & 0  & 1  & 0 & 0  \\
      0  & -1 & 0  & 0  & 0  & 1 & 0  \\
      0  & 0  & -1 & 0  & 0  & 0 & 1  \\
      0  & 0  & 0  & -1 & 0  & 0 & -1 \\
      0  & 0  & 0  & 0  & -1 & 1 & 0
    \end{pmatrix}
  \end{equation*}
\end{minipage}

\vspace{1em}
which yields,

\begin{equation*}
  (2\matr{I} + \matr{G})^{-1} = 0.354 \cdot \matr{I} + \begin{pmatrix}
    0     & -0.144 & -0.149 & 0.065  & 0.056  & -0.015 & -0.036 \\
    0.149 & 0      & -0.056 & -0.144 & 0.036  & 0.065  & 0.015  \\
    0.144 & -0.065 & 0      & 0.015  & -0.149 & -0.036 & 0.056  \\
    0.056 & 0.149  & -0.036 & 0      & -0.015 & -0.144 & -0.065 \\
    0.065 & -0.015 & 0.144  & 0.036  & 0      & 0.056  & -0.149 \\
    0.036 & 0.056  & 0.015  & 0.149  & 0.065  & 0      & 0.144  \\
    0.015 & -0.036 & 0.065  & -0.056 & 0.144  & -0.149 & 0      \\
  \end{pmatrix}
\end{equation*}
