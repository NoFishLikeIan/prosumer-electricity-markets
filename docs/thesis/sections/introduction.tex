\documentclass[../main.tex]{subfiles}

In recent years the economy found itself in the midst of a strong push towards decarbonisation of the energy sector. One aspect of such process is the increased adoption by households of technology to produce electricity independently, like solar panels, batteries, and ``smart'' meters (\cite{Parag2016}).

In coming years, this change in the households' role in the energy market, from consumers to prosumers, will present policy makers and economists with a vast spectrum of opportunities and challenges. In particular, the transition from centralized and oligopolistic fossil fuel producers to distributed renewable energy prosumers calls for a shift in economic modelling of energy markets, both in methodology and objectives. Such a system is prone to volatility, given prosumer's consumption decisions and the intermittent electricity production of

The project's main objective will be to identify the conditions under which a decentralized prosumer market for electricity displays aggregate resilience.

The research question will be dealt with on three dependent levels. First, the resilience properties will be investigated within a theoretical framework. Second, based on the theoretical model, an empirical analysis will be formulated. Third, the empirical result will define suitable policy implications.