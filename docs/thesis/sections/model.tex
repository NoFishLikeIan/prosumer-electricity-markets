\documentclass[../main.tex]{subfiles}
\section{Model}

The model focuses on two aspects of the cross-border electricity market. First, it models explicitly the bilateral trading procedure among electricity providers in the wholesale electricity market, which is a common mechanism of price setting, particularly in cross-border daily contracts in the EU and USA (\cite{Imran2014}). Doing this requires simplifying greatly the electricity market structure, particularly the different time scales at which operators trade, the intra-border competition, and the complex financial instruments. Nevertheless, these aspects can be approximated in my model by tweaking the network structure and these simplifications allows to concentrate on the bargaining-power induced by the graph, and the instabilities these might induce.

Providers set a \textit{local price} ($p_i$) at which they trade electricity with local market prosumers.

\begin{figure}[!ht]
    \centering
    \tikzstyle{prosumers} = [
draw,circle,
prosumer,
fill=prosumer,
text=black,
minimum size=5pt]

\tikzstyle{provider} = [
draw,circle,
minimum size=10pt]

\tikzstyle{market} = [
draw=gray, dashed, thick,
inner sep=8pt]


\begin{tikzpicture}[{Latex[scale=0.5]}-{Latex[scale=0.5]}, thick]

    % Placing providers
    \foreach \x/\y [count=\j] in {1/1, -1/1, 1/-1, -1/-1}
        % Draw provider
        {\node [provider]  (\j) at (2*\x, 2*\y) [fontscale=0.8] {\makecell[l]{Prov. \j \\ $X(p_{\j})$}};

            % Draw prosumers
            \edef\points{}
            \foreach \z/\w [count=\i] in {0/2, 1/2.1, 1.75/1.75, 2.1/1, 2/0}
                {\node [prosumers] (\x\y\i) at (\x + 1.3*\x*\z, \y + 1.5*\y*\w) [fontscale=0.6] {$x_{\i}$};
                    \path (\j) edge [] node [fontscale = 0.6] {} (\x\y\i);
                    \xdef\points{(\x\y\i) \points}
                };}

    \path
    (1) edge [-{Latex}] node [above, fontscale=0.8] {$Y^{(1, 2)}$} (2)
    (1) edge [-{Latex}] node [right, fontscale = 0.8] {$Y^{(1, 3)}$} (3)
    (1) edge [-{Latex}] node [above left, fontscale = 0.8] {$Y^{(1, 4)}$} (4)
    (3) edge [-{Latex}] node [above, fontscale = 0.8] {$Y^{(3, 4)}$} (4);

\end{tikzpicture}
    \caption{A stylized model structure}
    \label{tikz:model}
\end{figure}

