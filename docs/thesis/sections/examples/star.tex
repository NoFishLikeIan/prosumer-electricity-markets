A more interesting example is that of the star graph, namely each provider has as a neighbor provider 1. Assume there are 4 providers/markets as below.

\vspace{5mm}
\begin{minipage}{.5\textwidth}
    \resizebox{\textwidth}{!}{\tikzstyle{var} = [
draw,circle,
minimum size=10pt]

\tikzstyle{agent} = [
draw, circle,
minimum size=10pt]

\begin{tikzpicture}[-{Latex[scale=1]}, thick]

    \node [agent] (one) {Prov. $1$};
    \node [agent, left = 3cm of one] (two) {Prov. $2$};
    \node [agent, above = 3cm of one] (three) {Prov. $3$};
    \node [agent, right = 3cm of one] (four) {Prov. $4$};


    \path
    (one) edge [] node [above] {$Y^{(1, 2)}$} (two)
    (one) edge [] node [left] {$Y^{(1, 3)}$} (three)
    (one) edge [] node [above] {$Y^{(1, 4)}$} (four);

\end{tikzpicture}}
\end{minipage}
\begin{minipage}{.5\textwidth}
    \begin{equation*}
        \begin{split}
            \matr{A} &= \begin{pmatrix}
                0      & 1 & 1 & 1 & \ldots \\
                1      & 0 & 0 & 0 &        \\
                1      & 0 & 0 &   &        \\
                1      & 0 &   &   &        \\
                \vdots &   &   &   & \ddots
            \end{pmatrix} \\
            \vspace{5mm} \\
            E &= \set{(1, 2), (1, 3), (1, 4), \ldots, (1, n)} \\
            \vspace{5mm} \\
            \G &= \matr{\iota}_n \matr{\iota}_n^T - \I_n =  \begin{pmatrix}
                0 & 1 & 1 &        \\
                1 & 0 & 1 &        \\
                1 & 1 & 0 &        \\
                  &   &   & \ddots
            \end{pmatrix}
        \end{split}
    \end{equation*}
\end{minipage}
\vspace{5mm}

Given the associated line graph, consider the matrix,

\begin{equation*}
    \begin{split}
        (2\I_n + \G)^{-1} &= \left( 2\I_n + \matr{\iota}_n \matr{\iota}_n^T - \I_n \right)^{-1}\\
        &= \left( \I_n + \matr{\iota}_n \matr{\iota}_n^T \right)^{-1} \\
        &= \I_n - \frac{1}{n + 1} \matr{\iota}_n \matr{\iota}_n^T
    \end{split}
\end{equation*}

Then we can use this in the bargaining solution,

\begin{equation*}
    \begin{split}
        P_t \circ \Y_t &= (2\I_n + \G)^{-1} \Delta_t \\
        &= \left( \I_n - \frac{1}{n + 1} \matr{\iota}_n \matr{\iota}_n^T \right) \Delta_t \\
        &= \Delta_t - \frac{\sum_{(i, j)} \Delta^{(i, j)}_t}{n+1} \ \matr{\iota}_n
    \end{split}
\end{equation*}

Given the star structure of the network, I will write an edge $(1, j)$ as simply $j$ and I will suppress $t$. Hence, each cross-border price is,

\begin{equation*}
    P^j = \frac{n\Delta^j - \sum_{m \neq j} \Delta^m }{(n + 1)Y^j}.
\end{equation*}

Equalization of prices requires that, for each neighbor of $1$,

\begin{equation*}
    \begin{split}
        P^j &= P^i \\
        \frac{n\Delta^j - \sum_{m \neq j} \Delta^m }{Y^j} &= \frac{n\Delta^i - \sum_{l \neq i} \Delta^l}{Y^i} \\
        \frac{Y^i}{Y^j} &= \frac{\Delta^i - \sum_{l \neq i} \Delta^l / n}{\Delta^j - \sum_{m \neq j} \Delta^m / n}
    \end{split}
\end{equation*}

Then, using $X_1 = \sum^n_{j = 2} Y^j$, we can rewrite $Y^2$ as,

\begin{equation*}
    Y^2 = \sum^n_{l = 2} Y^l \Big/ \sum^n_{l = 2} \frac{Y^l}{Y^2}
\end{equation*}

Letting the average revenue difference between one and its neighbors be

\begin{equation*}
    \underline{\Delta}_1 = X_1 p_1 - \frac{1}{n-1} \sum^n_{l = 2} X_l p_l,
\end{equation*}

we can rewrite the cross-border prices as,

\begin{equation*}
    P^2 = \frac{n \Delta^2 - \sum_{m \neq 2} \Delta^m}{(n-1) Y^2} = \frac{\underline{\Delta}_1}{2 (n-1) X_1} = P^j \ \forall j.
\end{equation*}

Notice that if $n = 2$, we obtain back the solution of the two providers case.