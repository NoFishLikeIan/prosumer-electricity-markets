\documentclass[../thesis.tex]{subfiles}

\section{A appendix}

\subsection{Mathematical notation}

\begin{itemize}

    \item Matrices will be represented with bold characters, $\matr{A}$, and graphs with calligraphic characters, $\mathcal{A} = (V, E)$, where $V$ is the set of vertices and $E$ the set of edges.

    \item The operator $\diag$ sets all the off-diagonal entries to zero, \begin{equation*}
              \diag: \R^{n \times n} \to \R^{n \times n}, \ \ \diag(\matr{A}) = \begin{pmatrix}
                  a_{1, 1} & 0        & \ldots   &                   \\
                  0        & a_{2, 2} & 0        & \ldots &          \\
                  \vdots   & 0        & a_{3, 3} & 0      & \ldots   \\
                           &          & 0        & \ddots &        & \\
              \end{pmatrix}
          \end{equation*}

    \item I am using element-wise (Hadamard) operations between vertices and matrices extensively in the grid problem. The notation is the following, \begin{equation}
              \begin{split}
                  \text{With }x, y &\in \R^n \\
                  x \circ y &= \begin{pmatrix}
                      x_1 \cdot y_1 \\ x_2 \cdot y_2 \\ \vdots \\ x_n \cdot y_n
                  \end{pmatrix}, \ \
                  x \oslash y = \begin{pmatrix}
                      x_1 / y_1 \\ x_2 / y_2 \\ \vdots \\ x_n / y_n
                  \end{pmatrix}, \ \
                  x^{\circ m} = \begin{pmatrix}
                      x_1^m \\ x_2^m \\ \vdots \\ x_n^m
                  \end{pmatrix}
              \end{split}
          \end{equation}

\end{itemize}


\subsection{Network structure}

\begin{minipage}{0.45\textwidth}{
        \resizebox{\textwidth}{!}{
            \begin{tikzpicture}[-, thick]
                % Grids
                \node[main node] (1) {Grid$_1$};
                \node[main node] [below right = 2cm and 1cm of 1] (2) {Grid$_2$};
                \node[main node] [below left = 2cm and 1cm of 1] (3) {Grid$_3$};
                % Local markets 
                \node[main node] [above = 1cm of 1] (5) {Pros$_1$};
                \node[main node] [below right = 1cm and 1cm of 2] (6) {Pros$_2$};
                \node[main node] [below left = 1cm and 1cm of 3] (7) {Pros$_3$};
                % Paths
                \path[draw,thick]
                (1) edge node [above right] {$P, Y^{(1, 2)}$} (2)
                (2) edge node [below] {$P, Y^{(2, 3)}$} (3)
                (3) edge node [above left] {$P, Y^{(3, 1)}$} (1)
                (1) edge node [left] {$X(p_{1})$} (5)
                (2) edge node [above right] {$X(p_{2})$} (6)
                (3) edge node [above left] {$X(p_{3})$} (7);
            \end{tikzpicture}
        }}\end{minipage} \hfill
\begin{minipage}{0.5\textwidth}{
        The model is structured on two levels: local prosumer markets ($Pros_i$) and global electricity markets ($Grid_i$). The former are composed of heterogenous agents endowened with electricity and risk averse with respect to electricity consumption. The excess (defect) of energy generates a supply (demand) of electricity which is absorbed (met) by grid firms on the ``global'' energy market. These trade electricity with each other via bilateral bargaining.
    }
\end{minipage}

\subsection{Prosumers}

\subsubsection{Log utility}

The prosumer instantaneous utility of electricity consumption is,

\begin{equation*}
    u_i(x) = d \cdot \ln(x),
\end{equation*}

where $d$ is a scale parameter.

Each period, agents are endowed electricity, $e_{i, t}$, which follows a random exogenous process and have a given stock of cash-in-hand $m_{i, t}$. Furthermore agents can buy or sell electricity $x_{i, t}$ at price $p_t$. $x_{i, t} > 0$ implies that agents are buying electricity and $x_{i, t} < 0$ that they are selling it.

Suppressing $i$ for convenience, the dynamic optimization problem is then,

\begin{equation*}
    \begin{split}
        V(e_t) &= \sup_{x_t \in \R} \left\{u(x_t + e_t) + \beta \cdot \E_t V( e_{t+1} ) \right\} \\
        \\
        \textit{s.t. } m_{t+1} &= m_{t} - p_{t} \cdot x_{t}\\
        m_t  &\geq 0
        \\
        \textit{given } m_0&
    \end{split}
\end{equation*}

Agents are assumed to make forecasts over the price $p_{t+1}$, using a linear forecasting rule $\E_t[p_{t+1}] = \psi_h\cdot p_t + c$, where $c$ is a constant and $\psi_h \in \Psi$ is selected in a manner similar to \citeauthor{Hommes2013} (\citeyear{Hommes2013}).

Using the evolution of $m_t$ we can express the individual electricity demanded as,

\begin{equation}
    x_t = \frac{m_t - m_{t+1}}{p_t},
\end{equation}

such that the agents' control variable is the next level of cash-in-hand $m_{t+1}$. This yields the Euler equation,

\begin{equation}
    u^\prime\left( e_t + \frac{m_t - m_{t+1}}{p_t} \right) = \beta \cdot \E_t \left[ u^\prime\left(e_{t+1} + \frac{m_{t+1} - m_{t+2}}{ \psi_h \cdot p_t + c} \right)  \right].
\end{equation}

The endogenous grid method can then be used to numerically find a policy function

\begin{equation}
    m^\prime(m_t \ \vert \ \psi_h, p_t, e_t).
\end{equation}

The solution of the local problem yields the electricity demand given an endowment and different levels of $m_t$. The demand is plotted in Figure \ref{fig:demand}. Furthermore, in Figure \ref{fig:sim} I plot a simulation run of electricity demanded by an agent $x_t$ given an exogenous price mechanism.

\begin{figure}
    \centering
    \includegraphics[width=0.5\textwidth]{../../plots/markets/pricedemand.png}
    \caption{Electricity demand curve for an agent}
    \label{fig:demand}
\end{figure}

\begin{figure}
    \centering
    \includegraphics[width=0.5\textwidth]{../../plots/markets/simul.png}
    \caption{Simulation of an agent with switching forecasting rule and $AR(1)$ price process}
    \label{fig:sim}
\end{figure}



\subsubsection{Discrete aggregation}

Each prosumer market is composed of a discrete number of agents indexed by $i \in \set{1, 2, \ldots n}$. Each period the demand of agent $i$ is,

\begin{equation}
    x_{i, t}(p_t \ \vert \ e_t) = \frac{m_{i, t} - m^\prime(m_{i, t} \ \vert \ \psi_h, p_t, e_t).}{p_t}
\end{equation}

which yields aggregate demand,

\begin{equation}
    X_t(p_t) = \sum^n_{i = 1} x_{i, t}(p_t \ \vert \ e_t)
\end{equation}


\subsubsection{To do}

I would like the preferences to be Epstein-Zin, but could not find a solution to the problem at hand.

\subsection{Grid firms}

\subsubsection{Notation}

Grid firms do not face any intertemporal optimization. Each period they set the price on the local market $p$ and induce a demand (supply) $X(p)$. They then match $X(p)$ by trading with other grid firms quantities $Y$ at a bargained price $P$. Hence, hereafter I will suppress the time index $t$.

Grid firms operate on a graph $\mathcal{A} = (V, E)$. Firms are nodes, $i \in V$, and they can trade if they share an edge, $(i, j) \in E$. I will indicate the neighbors of a node as,

\begin{equation}
    N_{\mathcal{A}}(i) = \set{ \ j \in V: \ (i, j) \in E \ }
\end{equation}

\subsubsection{Setup}

Each period, the optimization problem of the firm is,

\begin{equation}
    \max_{p, Y} \Pi_i(p, Y)
\end{equation}

where,

\begin{equation}
    \Pi_i\left(p, Y\right) = X_i(p) \cdot p - \sum_{j \in N_{\mathcal{A}}(i)} Y^{(i, j)} \cdot P^{(i, j)}
\end{equation}

subject to,

\begin{equation}
    X_i=  \sum_{j \in N_{\mathcal{A}}(i)} Y^{(i, j)}
\end{equation}

For now assume $P^{(i, j)}$ is determined by (i.e. is a function of) the vector of traded quantities $Y$, with elements $Y^{(i, j)}$ for every $(i, j) \in E$. Suppressing $i$ for convenience (i.e. $Y^{(i, j)} \mapsto Y_j$), the Lagrangian is,

\begin{equation}
    \mathcal{L}\left(p, Y_j\right) = X(p) \cdot p - \sum_{j \in N_{\mathcal{A}}(i)} Y_j \cdot P_j (Y) + \lambda\cdot \left(X(p) - \sum_{j \in N_{\mathcal{A}}(i)} Y_j\right)
\end{equation}

with first order condition,

\begin{equation}
    \begin{split}
        \frac{\partial \mathcal{L}}{\partial p}  &= \frac{\partial X}{\partial p}(p) \cdot p + X(p) + \lambda \cdot \frac{\partial X}{\partial p}(p) = 0 \\
        \frac{\partial \mathcal{L}}{\partial Y^j}  &= P_j + \frac{\partial P_j}{\partial Y_j} \cdot Y_j + \lambda =0, \ \forall j \in N_{\mathcal{A}}(i)
    \end{split}
\end{equation}

Isolating the multiplier $\lambda$ from the first equation yields,

\begin{equation}
    - \lambda = p + \frac{X}{\partial X / \partial p}
\end{equation}

which, substituting in the second one, yields the system of equations, $\forall j \in N_{\mathcal{A}}(i)$,

\begin{equation} \label{firm_optimization}
    P_j(Y_j) + \frac{\partial P_j}{\partial Y_j} \cdot Y_j = p + \frac{X}{\partial X / \partial p}
\end{equation}

\subsubsection{Bargaining model}

To solve the bargaining model we need to impose a ``direction'' of trade, namely a traded quantity $Y^{(i, j)}$ enters positively in $j$ and negatively in $i$. To do this we can work with the directed graph $\mathcal{A}$ with upper triangular adjacency matrix $\matr{A}_{d}$.

To encode the asymmetry of trade hereafter I will work with the bi-directed graph with adjacency matrix,

\begin{equation}
    \matr{A} := \matr{A}_{d} - \matr{A}_{d}^T,  \ \matr{A} \in\R^{n\times n}.
\end{equation}

This matrix has entries $a_{i, j} \in \set{-1, 0, 1}$ with $a_{i, j} = - a_{j, i}$ and $a_{i, i} = 0$. For example, the graph of Section 1 would then be represented by,

\begin{equation}
    \matr{A}_d= \begin{pmatrix}
        0 & 1 & 1 \\
        0 & 0 & 1 \\
        0 & 0 & 0
    \end{pmatrix} \implies
    \matr{A} = \begin{pmatrix}
        0  & 1  & 1 \\
        -1 & 0  & 1 \\
        -1 & -1 & 0
    \end{pmatrix}
\end{equation}

The payoff of firm $i$ from the bargaining procedure can be written as,

\begin{equation}
    \begin{split}
        \Pi_i &= X_i(p_i)\cdot p_i - \sum_{j \in N_{\mathcal{A}}(i)} Y^{(i, j)} \cdot P^{(i, j)} \\
        &= X_i(p_i)\cdot p_i - \sum_{j \in V} a_{i, j} \cdot Y^{(i, j)} \cdot P^{(i, j)}
    \end{split}
\end{equation}

The Nash bargaining solution is such that,

\begin{equation}
    P^{(i, j)} = \arg \max_{P^{(i, j)}} \left\{\Pi_i \cdot \Pi_j \right\}.
\end{equation}

The first order condition is,

\begin{equation}
    \frac{\partial\Pi_i}{\partial P^{(i, j)}} \cdot \Pi_j + \frac{\partial\Pi_j}{\partial P^{(i, j)}} \cdot \Pi_i = 0.
\end{equation}

using,

\begin{equation}
    \frac{\partial\Pi_i}{\partial P^{(i, j)}} = - a_{i, j} \cdot Y^{(i, j)} = -\frac{\partial \Pi_j}{\partial P^{(i, j)}}
\end{equation}

we can rewrite the first order condition as,

\begin{equation} \label{foc_2}
    \begin{split}
        \Pi_i &= \Pi_j \\
        X_i(p_i)\cdot p_i - \sum_{m \in N} a_{i, m} \cdot Y^{(i, m)} \cdot P^{(i, m)} &= X_j(p_j)\cdot p_j - \sum_{m \in N} a_{j, m} \cdot Y^{(j, m)} \cdot P^{(j, m)}
    \end{split}
\end{equation}

Using $a_{i, j} = - a_{j, i}$, we can rewrite the two sums in (\ref{foc_2}) as,

\begin{equation}
    \begin{split}
        \sum_{m \in N} a_{i, m} \cdot Y^{(i, m)} \cdot P^{(i, m)} &= \sum_{m \in N \setminus \set{j}}  a_{i, m} \cdot Y^{(i, m)} \cdot P^{(i, m)} + a_{i, j} \cdot Y^{(i, j)} \cdot P^{(i, j)} \ \text{ and }\\
        \sum_{m \in N} a_{j, m} \cdot Y^{(j, m)} \cdot P^{(j, m)} &=  \sum_{m \in N \setminus \set{i}}  a_{j, m} \cdot Y^{(j, m)} \cdot P^{(j, m)} - a_{i, j} \cdot Y^{(i, j)} \cdot P^{(i, j)}
    \end{split}
\end{equation}

which yields, for every edge $(i, j)$ with $ a_{i, j} \neq 0$,

\begin{equation} \label{solution}
    \begin{split}
        P^{(i, j)} = \frac{1}{2\cdot Y^{(i, j)}} \Biggl( &\underbrace{X_i(p_i)\cdot p_i - X_j(p_j)\cdot p_j}_{\text{revenue difference }}
        \\  + &\underbrace{\sum_{m\in N\setminus \set{i}} a_{j, m} \cdot Y^{(j, m)} \cdot P^{(j, m)}}_{\text{outside option of } j}
        \\ - & \underbrace{\sum_{m \in N\setminus \set{j}} a_{i, m} \cdot Y^{(i, m)} \cdot P^{(i, m)}}_{\text{outside option of } i} \Biggr).
    \end{split}
\end{equation}

\subsubsection{Line graph and matrix notation}

% FIXME: This is the incidence matrix

To write Equation (\ref{solution}) in matrix notation, I introduce the line graph associated with our original graph. This is useful because all the vectors of interest, $P, Y, \Delta X$, are vectors over edges rather than nodes. A line graph is the graph with adjacency matrix $\matr{G}(\matr{A}) \in \R^{\abs{E} \times \abs{E}}$ were the nodes are edges of the original graphs and edges are inner nodes of the original graph

For example, consider the graph,

\vspace{0.5em}
\begin{minipage}{0.6\textwidth}
    \resizebox{\textwidth}{!}{
        \begin{tikzpicture}[{Latex[scale=1.25]}-{Latex[scale=1.25]}, thick]
            % Grids
            \node[main node] (1) {$1$};
            \node[main node] [right = 2cm of 1] (2) {$2$};
            \node[main node] [below right = 2cm and 2cm of 2] (5) {$5$};
            \node[main node] [above right = 2cm and 2cm of 2] (3) {$3$};
            \node[main node] [right = 2cm of 3] (4) {$4$};
            % Paths
            \path[draw,thick]
            (1) edge node [above] {$(1, 2)$} (2)
            (2) edge node [below left] {$(2, 5)$} (5)
            (2) edge node [above left] {$(2, 3)$} (3)
            (3) edge node [above] {$(3, 4)$} (4);
        \end{tikzpicture}}
\end{minipage} \hfill
\begin{minipage}{0.35\textwidth}
    \begin{equation*}
        \matr{A} = \begin{pmatrix}
            0  & 1  & 0  & 0 & 0 \\
            -1 & 0  & 1  & 0 & 1 \\
            0  & -1 & 0  & 1 & 0 \\
            0  & 0  & -1 & 0 & 0 \\
            0  & -1 & 0  & 0 & 0
        \end{pmatrix}
    \end{equation*}
\end{minipage}


The corresponding line graph is,

\vspace{0.5em}
\begin{minipage}{0.6\textwidth}
    \resizebox{\textwidth}{!}{
        \begin{tikzpicture}[{Latex[scale=1.25]}-{Latex[scale=1.25]}, thick]
            % Grids
            \node[main node] (12) {$(1, 2)$};
            \node[main node] [below right = 1cm and 2cm of 12] (25) {$(2, 5)$};
            \node[main node] [above right = 1cm and 2cm of 12] (23) {$(2, 3)$};
            \node[main node] [right = 2cm of 23] (34) {$(3, 4)$};
            % Paths
            \path[draw,thick]
            (12) edge node [below left] {$2$} (25)
            (12) edge node [above left] {$2$} (23)
            (23) edge node [above] {$3$} (34);
        \end{tikzpicture}}
\end{minipage} \hfill
\begin{minipage}{0.35\textwidth}
    \begin{equation*}
        \matr{G}(\matr{A}) = \begin{pmatrix}
            0   & 1 & 1   & 0 \\
            - 1 & 0 & 0   & 0 \\
            - 1 & 0 & 0   & 1 \\
            0   & 0 & - 1 & 0
        \end{pmatrix}
    \end{equation*}
\end{minipage}




\subsubsection{Line graph and bargaining solution}

In the bargaining problem we have a set of directed edges $E = \set{i \to j: i, j \in N}$. On each edge a bargaining price and a transferred quantity is determined. This defines the two vectors $P, \ Y \in \R^{\abs{E}}$. Furthermore let $\Delta X$ in $\R^{\abs{E}}$ be the vector of revenue difference with entries,

\begin{equation}
    \Delta X^{(i, j)} = X_i(p_i) \cdot p_i - X_{j}(p_j) \cdot p_j
\end{equation}

Equation (\ref{solution}) can be then written as,

\begin{equation} \label{matrix_solution}
    \begin{split}
        2(P \circ Y) &= \Delta X - \matr{G} \left( P \circ Y \right) \\
        (2\matr{I} + \matr{G}) (P \circ Y) &= \Delta X \\
        (P \circ Y) &= (2\matr{I} + \matr{G})^{-1} \Delta X \\
        P &= (2\matr{I} + \matr{G})^{-1} (\Delta X \oslash Y)
    \end{split}
\end{equation}

where $\circ$ and $\oslash$ denote the element-wise (Hadamard) product and division respectively. Using the Neumann expansion of $(2\matr{I} + \matr{G})^{-1}$ we obtain,

% FIXME: see Appendix \ref{a:neumann}

\begin{equation} \label{neumann_matrix_solution}
    P = \sum^{\infty}_{k=0}\frac{(-1)^k}{2^{k+1}} \matr{G}^k (\Delta X \oslash Y)
\end{equation}

\subsubsection{Solution of the bargaining problem}

Note that (\ref{neumann_matrix_solution}) defines the bargaining price as a function of the traded quantity $P(Y)$. Let $\partial P$ be the vector of derivatives $\partial P^{(i, j)} / \partial Y^{(i, j)}$ (i.e. the diagonal of the Jacobian, $J_P$).

\begin{equation}
    \partial P = - \diag\left(\sum^\infty_{i=0} \frac{(-1)^k}{2^{k+1}} \matr{G}^k\right) \Delta X \oslash Y^{\circ 2}
\end{equation}

We can use this in equation (\ref{firm_optimization}),

\begin{equation} \label{foc_grid}
    \begin{split}
        P + \partial P \circ Y &= \sum^{\infty}_{k=0}\frac{(-1)^k}{2^{k+1}} \matr{G}^k (\Delta X \oslash Y) - \diag\left(\sum^\infty_{i=0} \frac{(-1)^k}{2^{k+1}} \matr{G}^k\right) (\Delta X \oslash Y^{\circ 2}) \circ Y \\
        &= \underbrace{\left[ \sum^{\infty}_{k=0}\frac{(-1)^k}{2^{k+1}} \matr{G}^k - \diag \left( \sum^{\infty}_{k=0}\frac{(-1)^k}{2^{k+1}} \matr{G}^k \right) \right]}_{\matr{H(G)}} (\Delta X \oslash Y)
    \end{split}
\end{equation}

The matrix $\matr{H(G)}$ is an hollow matrix by construction (i.e. $\matr{H(G)}_{i, i} = 0 \ \forall i$).

\subsubsection{Equilibrium}

\subsubsection{To do}

Find a way to link equation $P + \partial P \circ Y$ with the optimization problem of the grid firm. It seems like $(P + \partial P \circ Y)_j = p + \frac{X}{\partial X / \partial p}$ for all $j$ does not allow for much of a solution. Furthermore, there is a nice economic interpretation for the equation for $P$ that has to do with the cycles of the matrix $\matr{G}$ (indirect effects of having an outside edge) which I have seen in some paper that I cannot find at the moment.

