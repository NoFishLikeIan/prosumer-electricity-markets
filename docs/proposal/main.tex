\documentclass[american]{scrartcl}

    \newcommand{\lang}{en}

    \usepackage{babel}
    \usepackage[utf8]{inputenc} 
    \usepackage{csquotes}
    \usepackage{amsmath}
    \usepackage{graphicx}    

    \setlength{\parindent}{0em}
    \setlength{\parskip}{0.5em}


    \usepackage[
        bibencoding=utf8, 
        style=apa
    ]{biblatex}

    \bibliography{refs}
    
    
    \usepackage{amsmath}
    \title{
        Trophic Analysis of a Prosumers Electricity Market
    }

    \author{Andrea Titton}
    
\begin{document}

\nocite{*}
\maketitle

\section{Introduction}

\section{Research Question}

The project's main objective will be to identify the conditions under which a decentralized market for electricity displays aggregate resilience.

The research question will be dealt with on three dependent levels. First, the resilience properties will be investigated within a theoretical framework. Second, based on the theoretical model, an empirical analysis will be formulated. Third, the empirical result will define the policy implications.

\section{Methodology}

\subsection{Notion of resilience}

The notion of network resilience will be mapped to three properties of dynamical systems: systemic stability, self-organized critical dynamics, and Lyapunov stability.

The first arises when a network's critical points are robust to perturbations of its nodes. For example, a desirable property of an electricity market of prosumers is its ability to prevent systematic shortages after a negative shock to one of the prosumer's production capacity due to a climate shock. The second, as presented by \citeauthor{Bak1995} (\citeyear{Bak1995}), concerns the capacity of the dynamical system to display ``similar" critical points independently of time and parameter space. In this context, the pricing mechanism and equilibria of the proposed market should be scale-free. The third property refers to the stability of a solution near the equilibrium. Taking again a prosumers electricity market as example, it is desirable for the price vector to be stable around the market clearing price.

\subsection{Theoretical model}

% Heterogenous agents, strategic decision every periods, network -> difference equation -> stability via trophic analysis -> empirical measures

The electricity market will be modeled as an exogenous cyclical network, where each node is a prosumer and each edge represents the possibility of trading electricity between two nodes (or of storing electricity at a cost, if the edge's source is the same as its target). The exogenous network will be based on \citeauthor{Brown2019} (\citeyear{Brown2019}).

Prosumers are assumed to be heterogenous, risk adverse agents with bounded rationality, making decisions in discrete time. Heterogeneity will be modeled over two dimensions: agents will have different degrees of risk tolerance and use different price forecasting rules. The former yields a model that allows for a price system were less risk adverse users bare more risk and absorb the shock uncertainty for other users. The latter, allows for herd behavior. % FIXME: This sentence should be elaborated further 

The agents behavior will be driven by a dynamic optimization problem in discrete time. In particular agents will maximize profit given current price, expected price dynamics, and an inelastic demand of energy. The two heterogeneities can be modeled in this framework by using Epstein-Zin preferences over the Bellman equation.

Furthermore, two pricing mechanisms will be considered on the market. First, an exogenous centralized energy price that clears aggregate demand and supply, which will not be efficient considering the network structure of the market. Second, a vector of prices (one for each edge) that solves the bargaining problems between two nodes, derived as in \citeauthor{Bedayo2016} (\citeyear{Bedayo2016}).

Given the structure of the model, the goal is to derive a law of motion (i.e.non-linear difference equation) for electricity production and analyze its dynamical properties. In particular, the focus will be on finding the combination of parameters space, pricing mechanism, and network structure that yields the optimal resilience, as defined above.

\iffalse
    \begin{figure}[h!]
        \centering
        \includegraphics[width=\linewidth,height=0.4\textheight,keepaspectratio]{../../plots/presentations/electricity.png}
        \label{fig:random_example}
        \caption{A potential realization of the described model}
    \end{figure}
\fi

\subsection{Trophic Analysis}

To simplify the problem space and ease the empirical testability of the model, we will use techniques from trophic analysis.

The goal is to create a link between the models structure and its dynamics, analogous to \citeauthor{MacKay2020} (\citeyear[p.~19]{MacKay2020}), by deriving a formula for the trophic levels and associated trophic incoherence value ($F_0$), defining a set of shocks the network could face, and then expressing the resilience to such shocks as a function of $F_0$.

Using trophic coherence as a pivot to the model's dynamics, as opposed to more traditional economic metrics such as upstreamness (\cite{Antrs2012}), offers three main advantages. First, it does not require ``source'' nodes in directed graphs hence it is defined for cycle graphs. Second, it is robust to local computation, that is, assuming we can only observe or model a subgraph, the trophic levels are constant in the inner graph and do not vary drastically around truncations (\cite[p.~19]{MacKay2020}). Third, its interpretation is fundamentally linked to that of spectral radius of a graph, which regulates the behavior of most dynamical systems on graphs, despite being much easier to measure. % FIXME: Is it clear what I mean here?

These three properties allow for an empirically and theoretically consistent, robust, and easily computable metric, that acts as a window on the inner workings of the network's dynamics. % FIXME: Weird phrasing maybe?


\subsection{Policy implications}



\newpage

\printbibliography
\end{document}