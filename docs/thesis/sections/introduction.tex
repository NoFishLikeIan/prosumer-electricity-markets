\documentclass[../thesis.tex]{subfiles}
\section{Introduction}

\iffalse notes

    - Decoupling: generators, providers, TSO
    - Providers coordination
    - Trade website in europe \href{https://www.smard.de/page/en/wiki-article/5884/6012#:~:text=Electricity%20is%20sold%20on%20the,the%20cross%2Dborder%20electricity%20trade}.

\fi

\subsection{Motivation}

In recent years the economy found itself in the midst of a strong push towards decarbonisation of the energy sector. One aspect of such process is the increased adoption by households of technology to produce electricity independently, like solar panels, batteries, and ``smart'' meters. This shift in the households' role not only increases volatility demand but introduces a spatial component in such volatility, which requires one to focus both on wholesale regional electricity markets and cross border electricity markets.

% TODO: Should I expand on how the model looks like? 

Prosumers introduce price volatility, via demand, due to the intermittence in electricity production of renewable sources (\cite{Rintamaeki2017}). This can generate local electricity imbalances for two reasons. From a technical stand point, electricity might not be able to move between backbone grids due to cables capacity or regulation asymmetries between transmission system operators (TSOs). From an economic stand point, there might be failure to trade electricity due to weak integration of cross-border markets, market concentration among electricity providers, or asymmetries in the providers bargaining power given by different locations in the international electricity network. In this paper I focus on the latter concerns by developing a stylized model of the interaction between regional wholesale electricity markets and cross-border electricity markets. In particular, I focus on demand induced electricity price hikes and the potential contagion in the cross-border electricity markets.

\subsection{Preliminary results}

The model shows how bargaining power, under myopic expectations, can have a strong impact on sustaining price imbalances and inducing electricity shortages in the cross-border electricity market. Under a simple bargaining framework, the providers' position in the network allows them to capitalize on demand shortages across the network, increase profits, and induce high prices throughout cross-border markets.

\subsection{Literature review} % Previous literature TODO: Expand the literature review

% TODO: Add EU/US initiatives
A substantial amount of work has been done already to study the technical and legislative aspects of the introduction of renewable energy and prosumers in the current energy system. Consequently, a wealth of policy advice has been produced on aligning cross-border regulations and increasing connectivity of international electricity markets. In this paper, I will refrain from modelling such technical aspects, like design of local grids (\cite{Parag2016}) or TSOs regulatory frameworks aimed at integration of cross border markets (\cite{Pollitt2008}).

I will rather focus on giving a theoretical foundation to some empirical facts. In particular, \citein{Gebhardt2013} found that strong price imbalances persist between regional wholesale markets, even if highly integrated, such as in Europe. Under rational expectations and increasing integration of regional markets, electricity suppliers ought to arbitrage and equalize prices across regions. Hence, rational expectations as a theoretical basis might not be sufficient to explain price formation in cross border markets. Another analysis, carried out by \citein{Newbery2016}, lays down the path and benefits of European regional markets integration, but assumes complete integration and rational expectations.

The framework presented below then abandons rational expectations and auction formats in modelling cross-border price formation. Rather, it builds on agent-based modelling frameworks, summarized extensively by \citein{Weidlich2008}, by introducing a cross-border pricing mechanism based on pairwise bargaining and naive expectations. Explicitly modelling bilateral trading among electricity providers in the wholesale electricity markets is in itself an extreme assumption, especially for North America and Europe, but it allows to focus on a bargaining power as a potential channel of price imbalances. Furthermore, it still is a common mechanism of price setting, particularly in cross-border daily contracts (\cite{Imran2014}). The bargaining procedure is based on a exogenous version of that developed for endogenous networks by \citein{Bedayo2016}.

Overall such modelling choices require simplifying greatly the electricity market structure, particularly the different time scales at which operators trade, the intra-border competition, and the complex financial instruments. Nevertheless, these aspects can be still approximated by tweaking the network structure and simplifying them allows to concentrate on the spatial bargaining power and the instabilities this might induce.