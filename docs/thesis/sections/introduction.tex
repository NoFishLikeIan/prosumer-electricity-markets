\documentclass[../main.tex]{subfiles}
\section{Introduction}

\iffalse notes

    - Decoupling: generators, providers, TSO
    - Providers coordination
    - Trade website in europe \href{https://www.smard.de/page/en/wiki-article/5884/6012#:~:text=Electricity%20is%20sold%20on%20the,the%20cross%2Dborder%20electricity%20trade}.

\fi

In recent years the economy found itself in the midst of a strong push towards decarbonisation of the energy sector. One aspect of such process is the increased adoption by households of technology to produce electricity independently, like solar panels, batteries, and ``smart'' meters. This shift in the households' role not only increases volatility demand but introduces a spatial component in such volatility, which requires one to focus both on wholesale regional electricity markets and cross border electricity markets.

% TODO: Should I expand on how the model looks like? 

Prosumers introduce price volatility, via demand, due to the intermittence in electricity production of renewable sources (\cite{Rintamaeki2017}). This can generate local electricity imbalances for two reasons. From a technical stand point, electricity might not be able to move between backbone grids due to cables capacity or regulation asymmetries between transmission system operators (TSOs). From an economic stand point, there might be failure to trade electricity due to weak integration of cross-border markets, market concentration among electricity providers, or asymmetries in the providers bargaining power given by different locations in the international electricity network. In this paper I focus on the latter concerns by developing a stylized model of the interaction between regional wholesale electricity markets and cross-border electricity markets.

% TODO: Write results

\subsection{Literature review} % Previous literature TODO: Expand the literature review

% TODO: Add EU/US initiatives
A substantial amount of work has been done already to study the technical and legislative aspects of the introduction of renewable energy and prosumers in the current energy system. Consequently, a wealth of policy advice has been produced on aligning cross-border regulations and increasing connectivity of international electricity markets. In this paper, I will refrain from modelling aspects of wholesale electricity markets that have been studied extensively, such as short-term dynamics (\cite{Weidlich2008}), technical aspects of the local grids (\cite{Parag2016}), and TSOs regulatory framework options for renewable energy integration (\cite{Pollitt2008}).

I will rather focus on giving a theoretical foundation to some key facts that have been observed. \citein{Gebhardt2013}

% \citein{Gebhardt2013} or \citein{Newbery2016}

To do so, first I will improve on the commonly used constant electricity demand (\cite{Weidlich2008}). Second, I will model cross-border trade as a result of a bargaining procedure on a network (\cite{Bedayo2016}).

The model models explicitly the bilateral trading procedure among electricity providers in the wholesale electricity market, which is a common mechanism of price setting, particularly in cross-border daily contracts in the EU and USA (\cite{Imran2014}). Doing this requires simplifying greatly the electricity market structure, particularly the different time scales at which operators trade, the intra-border competition, and the complex financial instruments. Nevertheless, these aspects can be approximated in my model by tweaking the network structure and these simplifications allows to concentrate on the bargaining-power induced by the graph, and the instabilities these might induce.

