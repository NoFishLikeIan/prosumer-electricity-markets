\documentclass[../main.tex]{subfiles}
\section{Model}

The model focuses on two aspects of the cross-border electricity market. First, it models explicitly the bilateral trading procedure among electricity providers in the wholesale electricity market, which is a common mechanism of price setting, particularly in cross-border daily contracts in the EU and USA (\cite{Imran2014}). Doing this requires simplifying greatly the electricity market structure, particularly the different time scales at which operators trade, the intra-border competition, and the complex financial instruments. Nevertheless, these aspects can be approximated in my model by tweaking the network structure and these simplifications allows to concentrate on the bargaining-power induced by the graph, and the instabilities these might induce.

Providers set a \textit{local price} ($p_i$) at which they trade electricity with local market prosumers.

\begin{figure}[!ht]
    \centering
    \tikzstyle{var} = [
draw,circle,
prosumer,
fill=prosumer,
text=black,
minimum size=10pt]

\tikzstyle{agent} = [
draw, circle,
fill=blue!30,
minimum size=10pt]

\tikzstyle{onnode} = [
    draw, circle, fill=white
]

\tikzstyle{time} = [
draw=gray, rectangle,
dashed,
thick,
inner sep=5pt]

\tikzstyle{market} = [
draw=gray, circle,
dashed,
thick,
inner sep=5pt]

\begin{tikzpicture}[-{Latex}, thick, every text node part/.style={align=center, fontscale=0.8}]

    % Placing providers
    \foreach \x/\y [count=\j] in {1/1, -1/1, 1/-1, -1/-1}
        % Draw provider
        {
            \node [agent]  (\j) at (2*\x, 2*\y) [] {Provider \j};
            \node [market] (\j-mkt) at (5*\x, 5*\y) [] {Wholesale \\ market \j};
            \path 
                (\j-mkt) edge [bend right] node [onnode] {$X_{\j}$} (\j)
                (\j) edge [dashed, bend right] node [onnode, dashed] {$p_{\j}$} (\j-mkt)
            ;
        }

    \path
    (1) edge [] node [above] {$Y^{(1, 2)}$} (2)
    (1) edge [] node [right] {$Y^{(1, 3)}$} (3)
    (3) edge [] node [above] {$Y^{(3, 4)}$} (4);

\end{tikzpicture}
    \caption{A stylized model structure}
    \label{tikz:model}
\end{figure}

