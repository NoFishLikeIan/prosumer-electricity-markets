\documentclass[../thesis.tex]{subfiles}
\section{Contagion on the Model}

Now that the groundwork has been layed down, we can focus on the object of our inquiry: demand shocks and price contagions. As equation (\ref{local_p_comp}) shows, price contagion depends, via $P(\matr{Y})$, linearly on the matrix $(2\I + \G)^{-1}$. We can call this the ``bargaining power'' matrix, since it allocates the excess revenues, $\Delta_t$, among providers in the network. For example, in the case of a positive demand shock and an increase in the local price of electricity of a given node, the excess demand will be partially absorbed by all other nodes in the network which, in turn, causes a contagion of local price hikes. Hence, the spread of the price hikes depends on the bargaining power of nodes in the matrix. Sticking with our example, if $X_{i, t}$ increases suddenly, $\Delta^{(i, j)}_t$ increases, and the cross-border prices across the network increase by
\begin{equation*}
  (2\I + \G)^{-1}_{(k, l), (i, j)}
\end{equation*}

This implies that a provider with a stronger bargaining position reacts more strongly to price changes, captures higher revenue, which leads to higher prices in the local market. Hence a more equal distribution of bargaining power leads to lower revenues for providers and lower price hikes. To formalize this, consider the entry $(i, j)$ in $P(\Y_t)$

\begin{equation*}
  P^{(i, j)}_t = \frac{\sum_{(l, m) \in E} \Delta^{(l, m)}_t \  (2\I + \G)^{-1}_{(i, j), (l, m)}}{Y^{(i, j)}_t},
\end{equation*}

where $\sum_{(l, m) \in E}$ is a summation over the row $(i, j)$ of $(2\I + \G)^{-1}$. Now assume there is some demand shock in a node $k$. We would like to understand the effect this has on prices today, $P^{(i, j)}_t$. The derivative of the price today with respect to demand today yields,

\begin{equation*}
  \begin{split}
    \frac{\partial P^{(i, j)}_{t}}{\partial X_{k, t}}
    &= \frac{1}{Y^{(i, j)}_t} \left( p_{k, t} \  \sum_{(k, m) \in E} (2\I + \G)^{-1}_{(i, j), (k, m)} - p_{k, t} \  \sum_{(l, k) \in E} (2\I + \G)^{-1}_{(i, j), (l, k)} \right) - \frac{P^{(i, j)}_t}{ Y_t^{(i, j)}} \\
    &= \frac{p_{k, t}}{Y^{(i, j)}_t} \underbrace{\left(\sum_{(k, m) \in E} (2\I + \G)^{-1}_{(i, j), (k, m)} -  \sum_{(l, k) \in E} (2\I + \G)^{-1}_{(i, j), (l, k)} \right)}_{\text{bargaining power of $k$ on the network}} - \frac{P^{(i, j)}_t}{ Y_t^{(i, j)}}
  \end{split}
\end{equation*}

This bargaining power value gives a first order approximation of the effect on the network of a demand shock in a given node. In Figure \ref{fig:influence} I plotted the influence of each node on different graphs that will be used in the simulation. It is wise to stop here with the analytical work on an arbitrary graph $\mathcal{A}$ given the complex behavior of $\partial P^{(i, j)}_{t + 1} / \partial X_{k, t}$. In the next section I will look at contagion by picking some simple examples, working out the core structure and influence analytically, and then simulating the systems' behavior.

\begin{figure}[H]
  \centering
  % First row
  \begin{subfigure}[t]{.4\textwidth}
    \centering
    \includegraphics[width=\linewidth]{\bargpath/line.pdf}
    \caption{On a path graph} \label{fig:linepower}
  \end{subfigure}
  \hfill
  \begin{subfigure}[t]{.4\textwidth}
    \centering
    \includegraphics[width=\linewidth]{\bargpath/star.pdf}
    \caption{On a star graph} \label{fig:starpower}
  \end{subfigure}


  \medskip

  % Second row
  \centering
  \begin{subfigure}[t]{.4\textwidth}
    \centering
    \includegraphics[width=\linewidth]{\bargpath/binarytree.pdf}
    \caption{On a binary tree} \label{fig:btreepower}
  \end{subfigure}
  \caption{Bargaining power over the network} \label{fig:influence}
\end{figure}

\subsection{Two providers}

\subfile{sections/examples/twoproviders.tex}

\subsection{Star and Path}

\subfile{sections/examples/starpath.tex}

