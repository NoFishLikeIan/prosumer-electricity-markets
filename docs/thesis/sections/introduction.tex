\documentclass[../thesis.tex]{subfiles}
\section{Introduction}

\iffalse notes

    - Decoupling: generators, providers, TSO
    - Providers coordination
    - Trade website in europe \href{https://www.smard.de/page/en/wiki-article/5884/6012#:~:text=Electricity%20is%20sold%20on%20the,the%20cross%2Dborder%20electricity%20trade}.

\fi

In recent years energy markets have been in the midst of a strong push towards decarbonisation. One aspect of such a process is the increased adoption by households of technology to produce electricity independently, for example solar panels, batteries, and ``smart'' meters. This shift in the role of households not only increases volatility demand but introduces a spatial component in such volatility. In this paper, by means of a theoretical model, I give a stylized mechanism by which demand shocks induce local price hikes, how these are transmitted in other local markets, and how they can generate sustained price imbalances. To do so, I focus on two markets for electricity: wholesale regional (or local) markets and cross-border markets. The former are centralized, cover a geographical region, and are highly integrated (i.e. they share regulators, grid, and transmission system operators). In the latter parties from different local markets trade with each other and can vary in degree of integration (\cite{Report2019}). This framework can be then used to evaluate policy aimed at electricity market stabilization, especially when demand is highly volatile because of prosumers households.

Prosumers introduce price volatility due to the intermittence in electricity production of renewable sources (\cite{Rintamaeki2017}). In particular, a failure in the prosumers' electricity production induces an exogenous and unexpected demand increase in the local wholesale market. This can generate electricity imbalances between local markets for two reasons. From a technical standpoint, electricity might not be able to move between backbone grids due to cables capacity or regulation asymmetries between transmission system operators (\cite{Parag2016, Pollitt2008}). From an economic standpoint, there might be failure to trade electricity due to weak integration of cross-border markets, market concentration among electricity providers, or asymmetries in the providers bargaining power given by different locations in the international electricity network (\cite{Ortner2019}). In this paper I focus on the latter concerns by developing a stylized model of the interaction between regional wholesale and cross-border electricity markets in the presence of myopic economic agents and spatial bargaining power.

There has been a lot of empirical and theoretical analysis on the interaction between local and cross-border electricity markets. \citein{Gebhardt2013} and \citein{Bockers2014} found that strong price imbalances persist between regional wholesale markets even if they are highly integrated, such as in Europe. Under rational expectations and increasing integration of regional markets, electricity suppliers ought to arbitrage and equalize prices across regions. Hence, rational expectations as a theoretical basis might not be sufficient to explain price formation in cross border markets. Another empirical analysis, carried out by \citein{Newbery2016}, lays down the path and benefits of European regional markets integration, but assumes complete integration and rational expectations.

The framework presented below then abandons rational expectations in modelling cross-border price formation. Rather, it builds on agent-based modelling frameworks, summarized extensively by \citein{Weidlich2008}, by introducing a cross-border pricing mechanism based on pairwise bargaining and naive expectations. Electricity providers act then as market makers, setting prices and trading electricity bilaterally in the cross-border market. This assumption might seem unrealistic and restrictive, especially for North America and Europe which have wholesale cross-border markets, but it allows to isolate bargaining power as a potential channel of price imbalances. Furthermore, pairwise bargaining still is a common mechanism of price setting in cross-border daily contracts (\cite{Imran2014}).

To model the bargaining procedure on the network I develop a framework based on Nash bargaining, as formulated by \citein{Rubinstein1992}. This formulation, extended to a network, draws an explicit equivalence between a provider's outside options and his bargaining power, which is a key feature in understanding transmission mechanism between regional and cross-border markets. On the other hand, it restricts the study to acyclic networks since the bargaining solution is underdetermined for arbitrary graphs with cycles. A different bargaining procedure such as that used in \citein{Bedayo2016}, based on an axiomatic approach, or in \citein{Corominas-Bosch2004}, focusing on bipartite graphs of buyers and sellers, would allow to study arbitrary graphs but makes it more complicated to link conditions in local market, outside options, and bargaining power. This increased complexity is the main motivation to forgo more sophisticated bargaining procedure in the model presented here.

The model shows how concentrated bargaining power induces sustained price imbalances between regions and electricity shortages in the cross-border electricity market. Under a simple bargaining framework, the providers' position in the network allows them to capitalize on demand shortages across the network, increase profits, and induce high prices throughout cross-border markets. Policy-wise, I find that in this framework price caps shield prosumers from price hikes but can induce blackouts. On the other hand, further integration of electricity markets reduces the likelihood of a system blackouts but, if not paired with reduction of bargaining power, it increases price imbalance and volatility.

Overall this ``artifact'' model structure abstracts away many key features of electricity markets, particularly the different time scales at which operators trade, the intra-border competition, and the use of complex financial instruments. Nevertheless, simplifying these aspects allows to concentrate on the spatial bargaining power and the localized demand shock, and the instabilities these might induce (\cite{Lopes2011}).

\iffalse
    A substantial amount of work has been done already to study the technical and legislative aspects of the introduction of renewable energy and prosumers in the current energy system. Consequently, a wealth of policy advice has been produced on aligning cross-border regulations and increasing connectivity of international electricity markets. In this paper, I will refrain from modelling such technical aspects, like design of local grids (\cite{Parag2016}) or TSOs regulatory frameworks aimed at integration of cross border markets (\cite{Pollitt2008}).
\fi