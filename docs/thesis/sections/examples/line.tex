\documentclass[../main.tex]{subfiles}
\subsection{Line}

Consider the case in which the network is line graph with $v$ nodes. Then $A$ is a $\R^{v \times v}$ matrix and $\matr{G}
$

\begin{equation}
    \begin{split}
        A_{i, j} = A_{j, i} &= \begin{cases}
            1 & \text{if } i = j - 1 \\
            0 & \text{otherwise}
        \end{cases} \\
        \matr{G}_{i, j} &= \begin{cases}
            1  & \text{if } i = j - 1 \\
            -1 & \text{if } i = j + 1 \\
            0  & \text{otherwise}
        \end{cases}
    \end{split}
\end{equation}

Note that this implies that,

\begin{equation}
    \matr{2I + G} =
    \begin{pNiceMatrix}[columns-width=auto]
        {2}  & {-1}  & {}       & {}       & {}  \\
        {1} & {2}  & {-1}      & {}       & {}  \\
        {}   & {1} & {\ddots} & {\ddots} & {}  \\
        {}   & {}   & {\ddots} & {\ddots} & {-1} \\
        {}   & {}   & {}       & {1}     & {2}
    \end{pNiceMatrix}
\end{equation}

is a tridiagonal Toeplitz matrix. This can be inverted using \citein{Huang1997},

\begin{equation}
    (\matr{2I + G})^{-1}_{i, j} = \begin{cases}
        (-1)^{i + j} \phi_{j + 1} \frac{\theta_{i - 1}}{\theta_n} & \text{if } i \leq j \\
        (-1)^{2i } \phi_{i + 1} \frac{\theta_{j - 1}}{\theta_n}   & \text{if } i > j
    \end{cases}
\end{equation}

Where,

\begin{equation}
    \begin{split}
        \phi_i &= 2\phi_{i + 1} + \phi_{i + 2} \ \text{ with } \ \phi_{n+1} = 1, \phi_n = 2 \\
        \theta_i &= 2\theta_{i - 1} + \theta_{i - 2} \ \text{ with } \ \theta_{0} = 1, \theta_1 = 2
    \end{split}
\end{equation}


$\theta_i$ is the sequence of shifted Pell numbers $P_{i+1}$ that have a closed form solution. Likewise, $\phi_i$ is the sequence of shifted Penn number $P_{n - i + 2}$. Hence we can rewrite,

\begin{equation}
    (\matr{2I + G})^{-1}_{i, j} = \begin{cases}
        (-1)^{i + j} \cdot P_{n + 1- j} \cdot P_i / P_{n+1} & \text{if } i \leq j \\
        (-1)^{2i } \cdot P_{n + 1- i} \cdot P_j / P_{n+1}   & \text{if } i > j
    \end{cases}
\end{equation}

Letting,

\begin{equation*}
    H(i, j, n) \coloneqq \frac{P_{n + 1 - j} \cdot P_i}{P_{n+1}}
\end{equation*}

we can rewrite,

\begin{equation}
    (\matr{2I + G})^{-1}_{i, j} = \begin{cases}
        (-1)^{i + j} H(i, j, n) & \text{if } i \leq j \\
        (-1)^{2i } H(j, i, n)   & \text{if } i > j
    \end{cases}
\end{equation}

Let the silver ratios be,

\begin{equation}
    \begin{split}
        \delta_{p} &= 1 + \sqrt{2} \\
        \delta_{m} &= 1 - \sqrt{2}
    \end{split}
\end{equation}

Using the properties, $P_{-a} = (-1)^{a+1} P_a$, $P_{a+b} = P_{a} P_{b+1} + P_{a-1} P_b$, and some manipulation

\begin{equation}
    \begin{split}
        \lim_{n \xrightarrow{} \infty} H(i, j, n) &= (-1)^{j + 1}  \left(P_j \cdot \delta_{p} - P_{j+1} \right) P_i \\
        &= \frac{(-1)^j }{2\sqrt{2}} \cdot (\delta_{p}^i - \delta_{m}^i) \cdot \delta^j_{m}
    \end{split}
\end{equation}

