\documentclass[../main.tex]{subfiles}
\section{Model}

The model focuses on two aspects of the cross-border electricity market. First, it models explicitly the bilateral trading procedure among electricity providers in the wholesale electricity market, which is a common mechanism of price setting, particularly in cross-border daily contracts in the EU and USA (\cite{Imran2014}). Doing this requires simplifying greatly the electricity market structure, particularly the different time scales at which operators trade, the intra-border competition, and the complex financial instruments. Nevertheless, these aspects can be approximated in my model by tweaking the network structure and these simplifications allows to concentrate on the bargaining-power induced by the graph, and the instabilities these might induce.

\iffalse
    \begin{figure}[!ht]
        \centering
        \tikzstyle{var} = [
draw,circle,
prosumer,
fill=prosumer,
text=black,
minimum size=10pt]

\tikzstyle{agent} = [
draw, circle,
fill=blue!30,
minimum size=10pt]

\tikzstyle{onnode} = [
    draw, circle, fill=white
]

\tikzstyle{time} = [
draw=gray, rectangle,
dashed,
thick,
inner sep=5pt]

\tikzstyle{market} = [
draw=gray, circle,
dashed,
thick,
inner sep=5pt]

\begin{tikzpicture}[-{Latex}, thick, every text node part/.style={align=center, fontscale=0.8}]

    % Placing providers
    \foreach \x/\y [count=\j] in {1/1, -1/1, 1/-1, -1/-1}
        % Draw provider
        {
            \node [agent]  (\j) at (2*\x, 2*\y) [] {Provider \j};
            \node [market] (\j-mkt) at (5*\x, 5*\y) [] {Wholesale \\ market \j};
            \path 
                (\j-mkt) edge [bend right] node [onnode] {$X_{\j}$} (\j)
                (\j) edge [dashed, bend right] node [onnode, dashed] {$p_{\j}$} (\j-mkt)
            ;
        }

    \path
    (1) edge [] node [above] {$Y^{(1, 2)}$} (2)
    (1) edge [] node [right] {$Y^{(1, 3)}$} (3)
    (3) edge [] node [above] {$Y^{(3, 4)}$} (4);

\end{tikzpicture}
        \caption{A stylized model structure}
        \label{tikz:model}
    \end{figure}

    \begin{figure}[!ht]
        \centering
        \tikzstyle{var} = [
draw,circle,
prosumer,
fill=prosumer,
text=black,
minimum size=10pt]

\tikzstyle{agent} = [
draw, circle,
fill=yellow,
minimum size=10pt]

\tikzstyle{derived} = [
draw, circle, dashed,
minimum size=10pt]

\tikzstyle{time} = [
draw=gray, rectangle,
dashed,
thick,
inner sep=5pt]

\tikzstyle{market} = [
draw=gray, rectangle,
dashed,
thick,
inner sep=15pt]


\begin{tikzpicture}[-{Latex[scale=1]}, thick]

    \node [agent] (producer_prev) {$\text{Prod}_{t-1}$};
    \node [var, below right = 1cm and 5cm of producer_prev] (s) {\makecell[c]{Elect. \\ $s_t$}};
    \node [var, right = 2cm of s] (s_prime) {$s_{t+1}$};
    \node [derived, below = 1cm of s] (demand) {\makecell[c]{Demand \\ $X_t$}};
    \node [derived, left = 1cm of demand] (producer_payoff) {\makecell[c]{Prod. payoff \\ $\pi_t$}};
    \node [var, below = 1cm of demand] (price) {$p_t$};
    \node [agent, below right = 0.5cm and 1cm of price] (provider) {Prov.};
    \node [derived, right = 1cm of demand] (provider_payoff) {\makecell[c]{Prov. payoff \\ $\Pi_t$}};


    \path
    (s) edge [dotted] node [above] {$\gamma$} (s_prime)
    (producer_prev) edge [] node [var] (ramp) {$r_{t-1}$} (s)
    (provider) edge (price)
    (price) edge (demand)
    (demand) edge (producer_payoff)
    (demand) edge (provider_payoff)
    (s) edge (demand)
    (producer_payoff) edge [-, bend left, dashed] node [fill=white] {$\E_{\psi}$}(producer_prev)

    (provider_payoff) edge [-, dashed] node [fill=white] {$\E$}(provider)

    (ramp) edge [] node [below left] {$c(r)$} (producer_payoff)
    ;

    \node [time, fit = (producer_prev) (ramp), label=above:{Previous period}] (Previous) {};
    \node [market, fit = (producer_prev) (s_prime) (s) (Previous), label=above:{Production}] (Production) {};
    \node [market, fit = (provider) (price), label=below:{Provider}] (Provider) {};

    \matrix [draw,below left, fill=white] at (current bounding box.north east) {
    \node [agent,label=right:{Agents}] {}; \\
    \node [derived,label=right:{Derived values}] {}; \\
    \node [var,label=right:{Variables}] {}; \\
    };

\end{tikzpicture}
        \caption{Model timing}
        \label{tikz:timing}
    \end{figure}
\fi

\subsection{Prosumers}

Prosumers are endowed an electricity quantity. The endowment follows a Markov process with some persistence $\rho$, $e_t \sim \matr{\pi}_{e}$, and a low and high state $\overline{e}$ and $\underline{e}$. Prosumers have a constant electricity consumption, the rest is sold. This can be encoded by making,

\begin{equation}
    \overline{e} > 0 \text{ and } \underline{e} < 0
\end{equation}


\subsection{Producers}

Electricity producers operate power plants and can generate electricity by ramping up or down production at a certain cost. This electricity is sold on the local market at a price $p$ given by electricity providers and costs $k$ to maintain. Furthermore, they assume the price formation mechanism to be independent of their production decisions and approximate it by a linear adapting forecasting rule, $p_{\psi, t+1} \cong \psi \cdot p_{t}$. Let $s$ be the electricity production and $c(s, r)$ the cost of ramping up or turning down production next period by $r$. I assume only a portion $\gamma$ of the electricity production can be freely reduced each period.This setup leads to the following optimization problem.

\begin{equation} \label{bellman_prod}
    \begin{split}
        V(s_t, p_t) &= \max_{r_t \in [-\gamma \cdot s_t, \infty)} \left\{ s_t \cdot (p_t - k) - c(s_t, r_t) \cdot r_t + \beta \cdot V(s_{t+1}, p_{t+1}) \right\} \\
        \text{ s.t. } s_{t+1} &= s_t + r_t \text{ with } s_t \geq 0, s_0 \\
        p_{t+1} &= \psi \cdot p_t
    \end{split}
\end{equation}

This optimization problem gives (see \ref{a:producer_optimization}) an implicit definition for $r$ as a function of $s$ and $p$ given $\psi$,

\begin{equation} \label{raw_policy_producer}
    \begin{split}
        r(s_t, p_t; \psi):& \\
        \frac{\partial c}{\partial r}(s_t, r) \cdot r + c(s_t, r) &= \beta \cdot \left\{  \psi \cdot p_t - k + \left[\frac{\partial c}{\partial r}(s_t + r, r) - \frac{\partial c}{\partial s_t}(s_t + r, r)\right] \cdot r + c(s_t + r, r) \right\} \\
        mc(r; s_t, p_t, \psi) &= \beta \cdot \{ mb(r; s_t, p_t, \psi) \}
    \end{split}
\end{equation}

The left hand side ($mc$) is the marginal cost today of ramping up production today. The producer by ramping up production pays $c(s_t, r)$ and increases the costs of ramping up by $\frac{\partial c}{\partial r}(s_t, r) \cdot r$. The right hand side ($mb$) is the marginal benefit tomorrow of ramping up production today. It can be broken up into the price the prosumer believes they can get for the new production, $\psi \cdot p_t$, the avoided marginal cost tomorrow of having increased production today instead, $\frac{\partial c}{\partial r}(s_t + r, r)  \cdot r + c(s_t + r, r)$, and finally the increased marginal cost brought about by a higher level of production $\frac{\partial c}{\partial s_t}(s_t + r, r)$.

\subsubsection{Quadratic costs}

For illustration purposes, consider the quadratic cost function $c(s, r) = r$ (see Figure \ref{fig:quadcosts}). This function allows us to explicitly derive,

\begin{equation}
    \begin{split}
        mc(r; s_t, p_t, \psi) &= 2r \\
        mb(r; s_t, p_t, \psi) &= \psi \cdot p_t - k + 2r
    \end{split}
\end{equation}

such that,

\begin{equation}
    r(s_t, p_t; \psi) = \frac{\beta}{2 (1 - \beta)}  \left(\psi \cdot p_t - k \right)
\end{equation}

\begin{figure}[!ht]
    \centering
    \includegraphics[width=\textwidth]{\plotpath/quadcosts.pdf}
    \caption{Quadratic cost function}
    \label{fig:quadcosts}
\end{figure}

\subsubsection{Softplus costs}

More realistically, ramping down should have no costs, namely $c(s, r) = 0$ if $r < 0$, and should be increasing both in scale of supply $s$ and in ramp-up level $r$. A differentiable function with (asymptotically) this feature is the softplus function (see Figure \ref{fig:costs}),

\begin{equation}
    c(s, r) = \frac{1}{c_1} \cdot \log \left( 1 + \exp(c_1 \cdot s \cdot r) \right)
\end{equation}

with partial derivatives,

\begin{equation}
    \frac{\partial c}{\partial s} = \frac{r}{1 + \exp( - c_1 \cdot s \cdot r)}  \hspace{5mm} \frac{\partial c}{\partial r} = \frac{s}{1 + \exp( - c_1 \cdot s \cdot r)}
\end{equation}

\begin{figure}[!ht]
    \centering
    \includegraphics[width=\textwidth]{\plotpath/cost.pdf}
    \caption{Softplus cost function}
    \label{fig:costs}
\end{figure}

The softplus function has the property that,

\begin{equation}
    \lim_{c_1 \xrightarrow{}\infty} c(s, r) = \max \{ 0, s \cdot r \}
\end{equation}

Using this property we can find an approximation of the policy function (see \ref{a:limiting}). Assuming positive unit profits, $\psi \cdot p_t > k$,

\begin{equation} \label{rpolicy}
    r(s_t, p_t; \psi) = \begin{cases}
        \frac{1 - \beta}{\beta} \cdot s_t - \frac{1}{2} \sqrt{ \left(\frac{1 - \beta}{\beta} \cdot 2 s_t \right)^2 - 4 \cdot \left(\psi \cdot p_t - k\right) } & \text{ if } s_t > \frac{\beta}{1-\beta} \sqrt{\psi \cdot p_t - k} \\
        \frac{\beta}{1-\beta} \sqrt{\psi \cdot p_t - k} - s_t                                                                                                  & \text{ otherwise }
    \end{cases}
\end{equation}

On the other hand, if $\psi \cdot p_t < k$, then,

\begin{equation}
    r(s_t, p_t; \psi) = -\gamma \cdot s_t.
\end{equation}

This function is plotted for two values of $\psi$ in Figure (\ref{fig:r}).

\begin{figure}[!ht]
    \centering
    \includegraphics[width=\textwidth]{\plotpath/rfunction.pdf}
    \caption{Ramp-up function with softplus costs}
    \label{fig:r}
\end{figure}

\subsubsection{Local demand}
Given the ramp-up policy function $r(s, p; \psi)$, it is possible to find the \textit{excess} demand of electricity in the local market, that is, the electricity demanded in the local market by local prosumers that it is not satisfied by local producers.

Assume that the local market is composed of $N$ producers and $M$ prosumers. First, let $R_t$ be the aggregated ramp-up function,

\begin{equation}
    R_{t}(p_t) \coloneqq \sum^N_{i = 1} r(s_{i, t}, p_t; \psi_{i, t})
\end{equation}

Then, the aggregate supply of electricity at time $t$, can be written as,

\begin{equation}
    \begin{split}
        \sum^N_{i = 1} s_{i, t+1} &= \sum^N_{i = 1} \left( s_{i, t} +  r(s_{i, t}, p_t; \psi_{i, t}) \right)\\
        S_{t+1} &= S_t + R_{t}(p_t)
    \end{split}
\end{equation}

The electricity demanded by $M$ prosumers is fully determined by the Markov chain of electricity endowments $e_t$. Furthermore, I assume that this demand is completely inelastic, namely prosumers will buy electricity at any price from providers or directly from producers. Hence the excess electricity demand at time $t$ is simply the difference between the demand and the supply,

\begin{equation} \label{x_inst}
    X_{t} = M \cdot e_{t} - S_{t}
\end{equation}

Note that if $S_t > M \cdot e_t$, there is an excess supply ($X_t < 0$) of electricity in the local market that the local provider can use to sell abroad. This happens in the case of a good endowment of electricity $e_t < 0$. Given equation (\ref{x_inst}), we obtain the evolution of the process $X_t$ (see \ref{a:ev_demand}),

\begin{equation} \label{x_true}
    X_{t+1} = X_t + M \cdot \left(e_{t+1} - e_t \right) -  R_t
\end{equation}

\subsubsection{Belief update}

Producers approximate the price evolution by assuming prices will change by a factor $\psi$ each period. I assume that this factor is picked from a set $\Psi$. Let $\psi_{t}$ be the ``forecasting factor'' currently employed by a producer. Next period, the producer obtains an instantenous payoff,

\begin{equation}
    \begin{split}
        s_{t+1} &= s_t + r(s_{i, t}, p_t; \psi_{t}) \\
        u_{t+1} &= s_{t+1} \cdot (p_{t+1} - k) - c(s_{t+1}, r_{t+1}) \cdot r(s_{t+1}, p_{t+1}; \psi_{t})
    \end{split}
\end{equation}

Furthermore, let $U_{\psi_h} \coloneqq \sum^t_{\tau = 0} \mathbbm{1} \{\psi_{\tau} = \psi_h\} \cdot u_{\tau}$, that is the cumulative sum of instantenous payoffs obtained while using factor $\psi_h$. Then producers update their forecasting rule by chossing the all time best performing factor, namely,

\begin{equation}
    \psi_{t+1} = \arg\max_{\psi_h} \frac{\exp(U_{\psi_h})}{\sum_{\psi \in \Psi} \exp(U_{\psi})}
\end{equation}


\subsection{Providers}

\subsubsection{Setup}

% FIXME: This paragraph

Electricity providers are monopolist in their own local market and are indexed, as is their local market, by $i$. They set the local price $p_{i, t}$ at which they sell (purchase) the excess demand (supply) of electricity in the local market $X_{i, t}$. On the one hand, if there is excess demand in the market, $X_{i, t} > 0$, providers sell electricity to prosumers at price $p_{i, t}$. On the other hand, if there is excess supply, $X_{i, t} < 0$, they buy electricity off of prosumers and producers at price $p_{i, t}$. The excess demand (supply) is satisfied (sold) by trading with other providers in the network. Hence, providers trade electricity $Y^{(i, j)}$, at a bargained price $P^{(i, j)}$, with neighboring providers. $Y^{(i, j)} > 0$ if $i$ is buying from $j$ and viceversa for $Y^{(i, j)} < 0$. This also implies that $Y^{(i, j)} = -Y^{(j, i)}$.Providers operate on a graph $\mathcal{A} = (V, E)$ and can trade only if they share an edge, $(i, j) \in E$. Let the neighbors of a node be the set,

\begin{equation}
    N_{\mathcal{A}}(i) \coloneqq \set{ \ j \in V: \ (i, j) \in E \ }
\end{equation}

Given this, each period, the optimization problem of the firm is,

\begin{equation}
    \max_{p_{i, t}, \Y_{i, t}} \Pi_i(p_{i, t}, \Y_{i, t})
\end{equation}

where $p_{i, t} \in \R_{+}$ and

\begin{equation}
    \begin{split}
        \Y_{i, t} &\in \R^{\abs{N_{\mathcal{A}}(i)}}, \hspace{5mm}\Y_{i, t} = \begin{pmatrix}
            Y^{(i, 1)}_t \\
            Y^{(i, 2)}_t \\
            \vdots
        \end{pmatrix}
    \end{split}
\end{equation}

is a vector of traded quantities with the provider's neighbors. Expanding the provider's payoff,

\begin{equation}
    \begin{split}
        \Pi_i(p_{i, t}, \Y_{i, t}) &= p_{i, t} \cdot X_{i, t}(p_{i, t-1}) - \sum_{j \in N_{\mathcal{A}}(i)} Y_t^{(i, j)} \cdot P_t^{(i, j)} \\
        \text{ s.t. } X_{i, t}(p_{i, t-1}) &=  \sum_{j \in N_{\mathcal{A}}(i)} Y_t^{(i, j)}
    \end{split}
\end{equation}

The condition $X_{i, t}(p_{i, t-1}) =  \sum_{j \in N_{\mathcal{A}}(i)} Y_t^{(i, j)}$ requires the provider to always match (sell) the quantity demanded (supplied) in the local market by trading with its neighbors. For now, assume every pairwise price on the network, $P^{(i, j)}$, is determined only by the vector of traded quantities, $\Y$. The provider's optimization is an intertemporal problem that depends on the state $X_t$ which follows the evolution layed down in equation (\ref{x_true}). The provider is assumed to make two simplifying assumptions on this evolution. First, at time $t$, it assumes $M \cdot (\E[e_{t+1}] - e_{t}) \cong 0$. This assumption is a good approximation in the case of a very persistent Markov chain $e_t$ (i.e. if $\matr{\pi}_e \approx \matr{I}$). Second, we assume the provider does not know the ramp-up function of producers in the local market, $r(s_t, p_t; \psi_t)$, hence they approximate the aggregate ramp-up function by a linear adaptive rule, proportional to the price, namely $R_t(p) \cong a_t + b_t \cdot p$ with $a < 0$ and $b > 0$.

Given this setup we can specify the optimization problem of the provider. Suppressing $i$ for convenience (i.e. $Y^j \coloneqq Y^{(i, j)}$), the Bellman equation can be written as,

\begin{equation*}
    \begin{split}
        V(X_t) &= \max_{p_t, \Y_t} \left\{ p_t \cdot X_t - \sum_{j \in N_{\mathcal{A}}(i)} Y^j_t \cdot P^j(\Y_t) + \lambda_t \cdot \left( X_t - \sum_{j \in N_{\mathcal{A}}(i)} Y^j_t \right) + \beta \cdot  V(X_{t+1}) \right\} \\
        X_{t+1} &= X_t - \left( a_t + b_t \cdot p_t \right)
    \end{split}
\end{equation*}

The optimization yields the policy function for local prices (\ref{a:provider_optimization}),

\begin{equation} \label{local_p}
    p_{t+1} = p_t + \frac{a_t}{b_t} +  \left( \frac{1 - \beta}{\beta} \right) \cdot \frac{X_t}{b_t} - \lambda_{t+1}
\end{equation}

\subsection{Bargaining model}

Given the evolution of the local price (\ref{local_p}), it is necessary to solve the bargaining model between providers (which determines the shadow price of acquiring outside electricity $\lambda_{t+1}$). To do so I first impose a ``direction'' of trade, namely a traded quantity $Y^{(i, j)}$ enters positively in $j$ and negatively in $i$, such that

\begin{equation}
    Y^{(i, j)} = -Y^{(j, i)}
\end{equation}

Furthermore we can rewrite the summation over the neighbors using $\matr{A}$, the adjacency matrix of the graph $\mathcal{A}$, with entries $a_{i, j}$, such that the instantaneous payoff of provider $i$ can be written as,

\begin{equation}
    \begin{split}
        \Pi_{i, t} &= X_{i, t} \cdot p_{i, t} - \sum_{j \in N_{\mathcal{A}}(i)} Y_t^{(i, j)} \cdot P_t^{(i, j)} \\
        &= X_{i, t} \cdot p_{i, t} - \sum_{j \in V} a_{i, j} \cdot Y_t^{(i, j)} \cdot P_t^{(i, j)}
    \end{split}
\end{equation}

I assume that the bargaining does not take into account individual market expectations and that it happens after the demand is already determined, such that $P$ is a function of $p$ only via $Y$. The Nash bargaining solution is such that,

\begin{equation}
    P_t^{(i, j)} = \arg \max_{P_t^{(i, j)}} \left\{\Pi_{i, t} \cdot \Pi_{j, t} \right\}.
\end{equation}


which yields (see \ref{a:barsol}), for every edge $(i, j)$ with $a_{i, j} \neq 0$,

\begin{equation} \label{bargaining_solution}
    \begin{split}
        P_t^{(i, j)} = \frac{1}{2\cdot Y_t^{(i, j)}} \Biggl( &\underbrace{X_{i, t} \cdot p_{i, t} - X_{j, t} \cdot p_{j, t}}_{\text{revenue difference }}
        \\  + &\underbrace{\sum_{m\in N\setminus \set{i}} a_{j, m} \cdot Y_t^{(j, m)} \cdot P_t^{(j, m)}}_{\text{outside option of } j}
        \\ - & \underbrace{\sum_{m \in N\setminus \set{j}} a_{i, m} \cdot Y_t^{(i, m)} \cdot P_t^{(i, m)}}_{\text{outside option of } i} \Biggr).
    \end{split}
\end{equation}


In the bargaining problem we have a set of directed edges $E = \set{i \to j: i, j \in N}$. On each edge a bargaining price and a transferred quantity is determined. This defines the two vectors $P, \ Y \in \R^{\abs{E}}$. One can therefore simplify the solution by using the directed line graph associated with the original graph, with adjacency matrix $\matr{G} \in \R^{\abs{E}\times \abs{E}}$. First, let $\Delta_t$ in $\R^{\abs{E}}$ be the vector of revenue difference in the respective local markets with entries,

\begin{equation}
    \Delta^{(i, j)}_t \coloneqq X_{i, t} \cdot p_{i, t} - X_{j, t} \cdot p_{j, t}
\end{equation}

At each time $t$, equation (\ref{bargaining_solution}) can be then written as,

\begin{equation} \label{matrix_bargaining_solution}
    \begin{split}
        2(P \circ \Y) &= \Delta  - \matr{G} \left( P \circ \Y \right) \\
        (2\matr{I} + \matr{G}) (P \circ \Y) &= \Delta  \\
        (P \circ \Y) &= (2\matr{I} + \matr{G})^{-1} \Delta \\
        \diag(\Y) P &=  (2\matr{I} + \matr{G})^{-1} \Delta \\
        P &= \diag(\Y)^{-1} (2\matr{I} + \matr{G})^{-1} \Delta
    \end{split}
\end{equation}

where $\circ$ denotes the element-wise (Hadamard) product. To get a better intuition for this result, we can use the Neumann expansion of $(2\matr{I} + \matr{G})^{-1}$,

\begin{equation}
    P = \diag(\Y)^{-1} \sum^{\infty}_{k=0}\frac{(-1)^k}{2^{k+1}} \matr{G}^k \Delta
\end{equation}

Hence, the bargaining solution gives the price following price formation,

\begin{equation}
    P(\Y_t) = \diag(\Y_t)^{-1} (2\matr{I} + \matr{G})^{-1} \Delta
\end{equation}

with associated Jacobian (see \ref{a:jacobian_p}),

\begin{equation}
    D_{\Y} P(\Y_t) = -\diag(P(\Y_t) \oslash \Y_t)
\end{equation}

We can use this, combined with the local optimization $\lambda_t$, to obtain an explicit expression of the local optimization. Using the fact that $D_{\Y} P(\Y_t)_{i, j} = 0$ if $i \neq j$, shadow price of the bargaining procedure is then (Equation \ref{lambda}),

letting,

\begin{equation}
    \delta(m) =\begin{cases}
        1  & \text{if } i > m \\
        -1 & \text{if } m < i
    \end{cases}
\end{equation}

\begin{equation}
    \begin{split}
        -\lambda_t &= P^m(\Y_t) + \sum_{j \in N_{\mathcal{A}}(i)} Y_t^j \cdot D_{\Y} P(\Y_t)_{j, m} \\
        &= \delta(m) \cdot 2 P^m(\Y_t)
    \end{split}
\end{equation}

Note that $\lambda_t$ needs to be the same for every neighbour, such that, for two neighbours $m$ and $n$

\begin{equation}
    \begin{split}
        \delta(m) \cdot 2 P^m(\Y_t) &= \delta(n) \cdot 2 P^n(\Y_t) \\
        \frac{P^n(\Y_t)}{P^m(\Y_t)} &= \frac{\delta(n)}{\delta(m)}
    \end{split}
\end{equation}


\subsubsection{Belief update}

Providers approximate the ramp-up production function of the local producers using a linear rule,

\begin{equation}
    R_{t+1}(p_t) \approx a_t + b_t \cdot p_t
\end{equation}

I assume that each period providers observe the realization $R_{t+1}(p_t)$ and pick the coefficients $a_{t+1}$ and $b_{t+1}$ via weighted least squares, more precisely

\begin{equation}
    \begin{pmatrix}
        a_{t+1} \\
        b_{t+1}
    \end{pmatrix} = (\matr{p}_{t}^T \matr{W} \matr{p}_{t})^{-1} (\matr{p}_{t}^T \matr{W} \matr{R}_{t+1})
\end{equation}

where,
\begin{equation}
    \matr{p}_{t} = \begin{pmatrix}
        1      & p_0    \\
        1      & p_1    \\
        \vdots & \vdots \\
        1      & p_{t}
    \end{pmatrix} \text{ and } \matr{p}_t = \begin{pmatrix}
        R_{1}  \\
        R_{2}  \\
        \vdots \\
        R_{t + 1}
    \end{pmatrix}
\end{equation}

and $\matr{W}$ is a weighting matrix of time exponential decay with entries,

\begin{equation}
    \matr{W}_{t_1, t_2} = \exp(-\alpha \cdot \abs{t_1 - t_2})
\end{equation}

To recap the belief structure of the model,

\begin{table}[h!]
    \centering
    \renewcommand{\arraystretch}{1.5}

\begin{tabular}{c  c | c}
  \headercell{Agent} & Actual process                                                                                                                              & Perceived process                                   \\
  \midrule
  \boxed{Provider}   & $R_t(p_t) = \sum^N_{i = 1} r(s_{i, t}, p_t; \psi_{i, t})$                                                                                   & $R_t(p_t) = N \cdot \left( b + a \cdot p_t \right)$ \\
                     & $M \cdot \left(e_{t+1} - e_t \right)$                                                                                                       & $0$                                                 \\
  \midrule
  \boxed{Producer}   & $p_{t+1}(p_t) = \lambda_{t+1} + \left( \frac{1 - \beta}{\beta} \right) \cdot \frac{X_t(p_{t-1})}{N \cdot a} - \frac{b +  a \cdot p_{t}}{a}$ & $p_{t+1}(p_t) = \psi \cdot p_t$
\end{tabular}
    \caption{Assumptions made by agents on other's processes}
    \label{table:perception}
\end{table}