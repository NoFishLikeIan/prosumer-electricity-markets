\documentclass[american]{scrartcl}

    \newcommand{\lang}{en}

    \usepackage{babel}
    \usepackage[utf8]{inputenc}     

    \setlength{\parindent}{0em}
    \setlength{\parskip}{0.5em}


    \usepackage[
        bibencoding=utf8, 
        style=apa
    ]{biblatex}

    \bibliography{refs}
    
    
    \usepackage{amsmath}
    \title{
        Trophic Model of Decentralized Electricity Markets
    }

    \author{Andrea Titton}
    
\begin{document}

\nocite{*}
\maketitle

\section{Introduction}

\section{Research Question}

The project's main objective is to identify the conditions under which a decentralized market for electricity displays aggregate resilience.

The research question will be dealt with on three dependent levels. First, the resilience properties will be investigated within a theoretical framework. Second, based on the theoretical model, an empirical analysis will be formulated. Third, the empirical result will define the policy implications.

\section{Methodology}

\subsection{Notion of resilience}

The notion of network resilience is mapped to three properties of dynamical systems: systemic stability, self-organized critical dynamics, and Lyapunov stability.

The first arises when a network's critical points are robust to perturbations of its nodes. For example, a desirable property of an electricity market of prosumers is its ability to prevent systematic shortages after a negative shock to one of the prosumer's production capacity due to a climate shock. The second, as presented by \citeauthor{Bak1995} (\citeyear{Bak1995}), concerns the capacity of the dynamical system to display ``similar" critical points independently of time, space, and parameter space. In this context, the pricing mechanism and equilibria of the proposed market should be scale-free. The third property refers to the stability of a solution near the equilibrium. Taking again a prosumers electricity market as example, it is desirable for the price vector to be stable around the market clearing price.

\subsection{Theoretical model}

% Heterogenous agents, strategic decision every periods, network -> difference equation -> stability via trophic analysis -> empirical measures

\subsection{Empirical Analysis}

\subsection{Policy implications}

\newpage

\printbibliography
\end{document}